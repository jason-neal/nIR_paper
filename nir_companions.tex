%%%%%%%%%%%%%%%%%%%%%%%%%%%%%%%%%%%%%%%%%%%%%%%%%%
% Basic setup. Most papers should leave these options alone.
\documentclass[fleqn,usenatbib]{mnras}

% MNRAS is set in Times font. If you don't have this installed (most LaTeX
% installations will be fine) or prefer the old Computer Modern fonts, comment
% out the following line
\usepackage{newtxtext,newtxmath}
% Depending on your LaTeX fonts installation, you might get better results with one of these:
%\usepackage{mathptmx}
%\usepackage{txfonts}

% Use vector fonts, so it zooms properly in on-screen viewing software
% Don't change these lines unless you know what you are doing
\usepackage[T1]{fontenc}
\usepackage{ae,aecompl}


%%%%% AUTHORS - PLACE YOUR OWN PACKAGES HERE %%%%%

% Only include extra packages if you really need them. Common packages are:
\usepackage{graphicx}	% Including figure files
\usepackage{amsmath}	% Advanced maths commands
%\usepackage{amssymb}	% Extra maths symbols
\usepackage[para, flushleft]{threeparttable}
\usepackage{hyperref}
\usepackage{natbib}
\usepackage{booktabs}
\usepackage{color}
\usepackage{fix-cm}
\usepackage{caption}   % Caption is required to avoid errors.
%%%%%%%%%%%%%%%%%%%%%%%%%%%%%%%%%%%%%%%%%%%%%%%%%%

%%%%% AUTHORS - PLACE YOUR OWN COMMANDS HERE %%%%%

% Please keep new commands to a minimum, and use \newcommand not \def to avoid
% overwriting existing commands. Example:
%\newcommand{\pcm}{\,cm$^{-2}$}	% per cm-squared
\newcommand{\kmps}{\,kms\(^{-1}\)}	% kilometers per second

\newcommand*\bl{\color{blue}}
\newcommand*\rd{\color{red}}
%%%%%%%%%%%%%%%%%%%%%%%%%%%%%%%%%%%%%%%%%%%%%%%%%%

%%%%%%%%%%%%%%%%%%% TITLE PAGE %%%%%%%%%%%%%%%%%%%

% Title of the paper, and the short title which is used in the headers.
% Keep the title short and informative.
%\title[Short title, max. 45 characters]{MNRAS \LaTeXe\ template -- title goes here}
\title[Towards the nIR detection of BD companions]{Towards the near-infrared detection of brown dwarf companions: Exploring methods to detect low mass stellar companions from blended spectra \thanks{Based on observations collected at the European Organisation for Astronomical Research in the Southern Hemisphere under ESO programme 089.C-0977(A)}}
%\subtitle{Exploring methods to detect low mass stellar companions from blended spectra}
% The list of authors, and the short list which is used in the headers.
% If you need two or more lines of authors, add an extra line using \newauthor
 
\author[J. J. Neal et al.]{
J. J. Neal$^{1,2}$\thanks{E-mail: jason.neal@astro.up.pt}
P. Figueira$^{3,1}$ 
N. C. Santos$^{1,2}$
C. Melo$^{3}$
\\
% List of institutions
$^{1}$Instituto de Astrof\'{\i}sica e Ci\^encias do Espa\c{c}o, CAUP, Universidade do Porto, Rua das Estrelas, PT4150-762 Porto, Portuga\\l
$^{2}$Departamento de F\'{\i}sica e Astronomia, Faculdade de Ci\^{e}ncias, Universidade do Porto, Portugal\\
$^{3}$European Southern Observatory, Alonso de C\'{o}rdova 3107, Vitacura, Casilla 19001, Santiago 19, Chile\\
}

% These dates will be filled out by the publisher
\date{Accepted XXX. Received YYY; in original form ZZZ}

% Enter the current year, for the copyright statements etc.
\pubyear{2018}

% Don't change these lines
\begin{document}
\label{firstpage}
\pagerange{\pageref{firstpage}--\pageref{lastpage}}
\maketitle

% Abstract of the paper
\begin{abstract}
   {\rd In this paper we attempt to directly detect the near-infrared spectrum of candidate brown dwarf (BD) companions around FGK stars to assert or discard their stellar nature. We explore two different methods to probe the faint spectra of BD companions. The first technique involves the direct subtraction of two observations shifted to mutually cancel out the spectra of the host star. The second technique applies a \(\chi^{2}\) fit to the individual observations of a synthetic binary model comprised of two PHOENIX-ACES spectra. The observed spectra are wavelength calibrated and corrected for the atmospheric absorption with the aid of synthetic telluric models. The direct subtraction method failed to detect the signature of the companions due to poor observational constraints. The \(\chi^{2}\) technique developed here is able to fit the component for the host star but unable to successfully detect the faint spectra of even the largest companion in our sample. We explore how the companion recovery fitting performs on simulated observations and discuss reasons for the non-detection observed. {\rd From the injection-recovery analysis, this technique in its current form, is insufficient to recover a companion below 3800~K which corresponds to an upper mass limit for the companions around \(\rm 600~M_{Jup}\)}. This work highlights the challenges in the spectral detection of faint companions. We explore the limitations of the direct subtraction method in the case of small RV separation, and companion detection with synthetic models at low companion/host flux ratios below 3\%.}
\end{abstract}

\begin{keywords}
brown dwarfs -- binaries: spectroscopic -- infrared: stars -- techniques: spectroscopic -- methods: miscellaneous
\end{keywords}


\section{Introduction}
\label{sec:intro}
Brown dwarfs (BDs) are sub-stellar objects unable to achieve hydrogen fusion, with masses around \(13-80~\textrm{M}_{jup} \)~\citep{chabrier_theory_2000}, bridging the gap between low-mass stars and giant planets. Without sustained fusion, brown-dwarfs cool down over time with an age-dependent cooling rate. Therefore, there is an inherent degeneracy between the mass, age and luminosity of a given BD\citep{burrows_nongray_1997}. This degeneracy may be broken by observation of several parameters, for instance when a BD is in a binary system with a main sequence host star, using both the host stars age and the masses derived from the dynamical motion.

A paucity of BD companions exists in short period orbits around Sun-like stars (\(\lesssim5 \)\,AU), compared to stellar or planetary companions, termed the \emph{brown dwarf desert}~\citep{halbwachs_exploring_2000,zucker_analysis_2001,sahlmann_search_2011}. As the number of known BDs orbiting solar type stars is low, the characterization of benchmark BDs in the brown dwarf desert~\citep[e.g.][]{crepp_trends_2016} is beneficial in understanding this sub-stellar population and to help constrain formation and evolution theories~\citep{whitworth_formation_2007}. The BD desert also provides a greater challenge as it reduces the amount of good BD candidates to study.

BDs in binary systems, unlike free-floating BDs, allow for the determination of their masses, when complemented with radial velocity (RV) and astrometry measurements. The RV technique provides the mass lower-limit (\(\textrm{M}_{2}\sin{i} \)) of binary and planetary companions, while complementary astrometry measurements can often provide mass upper-limits~\citep[e.g.][]{sahlmann_search_2011}. Measuring or tightening the constraints of BD masses improves the understanding of mass dependence on BD formation processes. For instance, there is growing evidence that the larger giant planets and BD companions do not follow the well known metallicity-giant planet correlation seen in main-sequence stars with planets~\citep[e.g.][]{santos_spectroscopic_2004,santos_observational_2017, maldonado_searching_2017}. Photometry along with stellar evolution models~\citep[e.g.][]{baraffe_evolutionary_2003,allard_btsettl_2013} can also be used to estimate the mass of BD companions~\citep[e.g.][]{moutou_eccentricity_2017} if there is sufficient orbital separation, and a precise determination of the age~\citep{soderblom_ages_2010}. 

Recently, there has been a renewed interest in BD candidates triggered by exoplanetary searches. While several works found similar properties on the two populations, like a similar density~\citep{hatzes_definition_2015}, others found intriguing differences. One of the most recent is the different host metallicity of the Brown Dwarf and giant planet populations~\citep{santos_observational_2017, schlaufman_evidence_2018}, a very strong hint of different formation mechanisms.

Spectral observations of binary systems contain the spectra of both bodies, in proportion to their flux ratio, and Doppler shifted relative to each other due to their orbital motion. One technique to recover the spectra of the companion is secondary reconstruction through a differential spectrum~\citep{ferluga_separating_1997}. Spectra from different phases are shifted to the host stars rest frame and subtracted to mutually cancel out the spectrum of the host star allowing the faint companion spectra to become visible. Advances in high-resolution and near-infrared (nIR) capabilities should enable this technique to be applied to BDs and planet companions, in which smaller RV shifts can be resolved and the contrast ratio of the smaller companion is improved. 

Observing in the nIR is specifically desirable for the cooler sub-stellar and giant planet companions as their thermal emission is stronger in the infrared compared to the optical. This improves the contrast ratio between the host star and companion, providing favourable conditions for their detection and spectral separation. CRIRES, a high resolution nIR spectrograph, has made many prominent advances in recent years with the detection of atmospheric constituents, such as \(\textrm{CO} \) and \(\textrm{H}_{2}\textrm{O} \), atmospheric winds and thermal profiles, rotation and orbital motion, for both transiting and non-transiting planets~\citep[e.g.][]{snellen_orbital_2010, brogi_signature_2012, rodler_weighing_2012, dekok_detection_2013, brogi_carbon_2014, snellen_fast_2014, piskorz_evidence_2016, brogi_rotation_2016, birkby_discovery_2017}.

The higher temperature and relatively larger size of BDs compared to giant-planets makes the development of spectral recovery techniques for BD companions a logical step towards the spectroscopic detection of planetary atmospheres. There has been the recent installation and continued development of many new high-resolution nIR spectrographs, such as, {CARMENES}~\citep{quirrenbach_carmenes_2014}, NIRPS~\citep{bouchy_nearinfrared_2017} or SPIRou~\citep{artigau_spirou_2014}, as well as, the CRIRES+~\citep{dorn_crires_2016} upgrade. These new instruments motivate the study of nIR-oriented methodologies for spectral recovery, and are of high importance due to the larger planet-to-star flux ratio provided by near-IR compared to the visible.

{\rd The search and detection of faint secondary spectra is not only relevant to planetary atmospheres. \citet{kolbl_detection_2015} developed a method to detect the presence of optical secondary spectra down to a flux ratio of 1\% in the hosts of \emph{Kepler} transit candidates. The presence of which can cause ambiguities in the system configuration, and increase the uncertainty of the measured planet radius. The characterization of the false positive probability rate for Kepler has been found to be as high as  \(\sim\)35\%~\citet{santerne_sophie_2012}.}

In this paper we apply two different techniques on FGK stars with BD companions with the aim to spectroscopically detect their companions. In Sect.~\ref{sec:data} we present the observations and reduction process as well as the spectral models used in this work. In Sect.~\ref{sec:spec_diff} we explain the differential spectral technique and its applicability to these observations while in Sect.~\ref{sec:results} we apply companion recovery using a \(\chi^{2} \) approach. In Sect.~\ref{sec:discussion} we discuss our results and in Sect.~\ref{sec:conclusions} we present our conclusions.


\section{Motivation and target selection}
\label{sec:motivation}
The work of~\citet{sahlmann_search_2011} identified several candidate brown dwarf companions of FGK stars with \(\rm M_2 \sin{i}\) values >\(\rm 10~M_{Jup}\). Seven candidates {\rd from~\citet{sahlmann_search_2011}, which were visible in Period 89,  were selected for  further} observation in order to identify their stellar nature. The target host stars are presented in Table~\ref{tab:starparams} with their stellar parameters, while the companion orbital parameters are provided in Table~\ref{tab:orbitparams}.

We note that some of these orbital parameters have been refined in the literature since observations took place. For example three candidates have had their masses refined in recent works. The companion to {HD 211847} was determined to be a low mass star with an \(\rm M_2=155~M_{Jup}\)~\citep{moutou_eccentricity_2017}, while the companion to {HD 4747} was found to have a mass of \(\rm M_2=60~M_{Jup}\)~\citep{crepp_trends_2016}. The two companions of {HD 202206} (B and c) were found to have masses of \(\rm M_B =93.6~M_{Jup}\) and \(\rm M_c = 17.9~M_{Jup}\), respectively, classifying {HD 202206}c as a ``circumbinary brown dwarf''~\citep{benedict_hd_2017}. These targets with recently refined masses create good benchmarks for us to compare any results of the techniques developed in this paper, and show that these masses do span the BD -- low mass star range. All target companions except {HD 162020} (P=8.4 days) are in (very) long period orbits (P=0.7--38 years) with masses (or \(M\sin{i}\)) greater than \(\rm 10~M_{Jup}\).

\textit{K}-band spectra were obtained with the goal to achieve a direct detection of the companion spectra through the application of a spectral-differential approach. Doing so would enable a further constraint to be placed on their masses. The \textit{K}-band is used to achieve a high contrast relative to the host star, detected in the extreme V-K colour indexes (>7.8). 

% Table of stellar parameters
%!TEX root = ../nir_companions.tex

%% Table of stellar parameters
\begin{table*}
        \small
        \centering
        \begin{threeparttable}[b]
   
        \caption{Stellar parameters of the target companion's host stars. V is the apparent magnitude taken from {SIMBAD}~\citep{wenger_simbad_2000}. {\rd Distances were calculated from the GAIA parallax measurements.}}
        \begin{tabular}{l c c r@{$~\pm~$}l r@{$~\pm~$}l r@{$~\pm~$}l r@{$~\pm~$}l c c c}
            \toprule
            Star & SpT & V &  \multicolumn{2}{c}{\(T_{\textrm{eff}}\) (K)} &  \multicolumn{2}{c}{logg (cm s\(^{-2} \))}  & \multicolumn{2}{c}{[Fe/H]} &  \multicolumn{2}{c}{\(M_1\) (M\(_{\sun} \))} & Age (Gyr) & d (pc) & Reference\\
            \midrule
            {HD 4747}     & K0V & 7.15 & 5316 & 50 & 4.48 & 0.10  & $-$0.21 & 0.05 & 0.81 & 0.02  & $3.3 \pm 2.3$   & $18.80 \pm 0.04$ & 1, 2, 3, 8 \\ 
            {HD 162020} & K3V & 9.12 & 4723 & 71 & 4.31 & 0.18  & $-$0.10 & 0.03 & 0.74 & 0.07  & $3.1 \pm 2.7$   & $30.85 \pm 0.06$ & 4, 5, 6, 8 \\  
            {HD 167665} & F9V & 6.40 & 6224 & 50 & 4.44 & 0.10  & 0.05       & 0.06 & 1.14 & 0.03  & 0.7 -- 3.6           & $ 31.24 \pm 0.06$ & 1, 8 \\
            {HD 168443} & G6V & 6.92 & 5617 & 35 & 4.22 & 0.05 & 0.06       & 0.05 & 1.01 & 0.07  & $10.0 \pm 0.3$  & $39.67 \pm 0.12$ & 5, 6, 8 \\ 
            {HD 202206} & G6V & 8.07 & 5757 & 25 & 4.47 & 0.03 & 0.29       & 0.02 & 1.04 & 0.07  & $2.9 \pm 1.0$    & $46.03 \pm 0.14$ & 5, 7, 8 \\ 
            {HD 211847} & G5V & 8.62 & 5715 & 24 & 4.49 & 0.05 & $-$0.08 & 0.02 & 0.92 & 0.07   & 0.1 -- 6.0            & $48.81 \pm 0.13 $ & 1, 2, 4, 8 \\ 
            {HD 30501}   & K2V & 7.59  & 5223 & 50 & 4.56 & 0.10 & 0.06       & 0.06 & 0.81 & 0.02  & 0.8 -- 7.0           & $20.37 \pm 0.01$ & 1, 4, 8 \\
            \bottomrule
        \end{tabular}
         \label{tab:starparams}
        \begin{tablenotes}
           \item[] References: (1)~\citet{sahlmann_search_2011}; (2)~\citet{santos_spectroscopic_2005}; (3)~\citet{crepp_trends_2016}; (4)~\citet{tsantaki_deriving_2013}; (5)~\cite{bonfanti_age_2016}; (6)~\citet{santos_spectroscopic_2004}; (7)~\citet{sousa_spectroscopic_2008}; {\rd(8)}~\citet{collaboration_gaia_2018};
         \end{tablenotes}
       \end{threeparttable}

\end{table*}

% Table of orbit parameters
%!TEX root = ../nir_companions.tex

% Table of orbit parameters
\begin{table*}
    \centering
    \footnotesize
    \begin{threeparttable}[b]
        \caption{Orbital parameters for the BD companions obtained from the literature.}
        \begin{tabular}{l c r@{$ \,\pm\, $}l r@{$ \,\pm\, $}l r@{$ \,\pm\, $}l r@{$ \,\pm\, $}l r@{$ \,\pm\, $}l cc c c}
            \toprule
            Object       & \(\gamma \)          & \multicolumn{2}{c}{Period} & \multicolumn{2}{c}{\(e \)} & \multicolumn{2}{c}{\(\textrm{K}_{1} \)} & \multicolumn{2}{c}{\(T_{0} \)}       & \multicolumn{2}{c}{\(\omega \)} & \(M_2\sin{i}\)     & \(M_2\)              & Ref.\\
                         & {\bl (kms\(^{-1}\))} & \multicolumn{2}{c}{(day)}  & \multicolumn{2}{c}{}       & \multicolumn{2}{c}{(ms\(^{-1} \))}      & \multicolumn{2}{c}{(JD-2\,450\,000)} &  \multicolumn{2}{c}{(deg)}      & \(\rm {M}_{Jup} \) &   \(\rm {M}_{Jup} \) & \\
            \midrule
            {HD 4747}    & $0.215 \pm 11 $   & 13826.2 & 314.1     & 0.740 & 0.002 & 755.3  & 12   & 463.1   & 7.3  & 269.1    & 0.6 &  39.6 & 60.2         & 1    \\
            {HD 162020}  & $-27.328\pm0.002$ & 8.42819 & $6e^{-5}$ & 0.277 & 0.002 & 1813   & 4    & 1990.68 & 0.01 & 28.4     & 0.2 & 14.4  & --           & 2    \\
            {HD 167665}  & $8.003 \pm 0.008$ & 4451.8  & 27.6      & 0.340 & 0.005 & 609.5  & 3.3  & 6987.6  & 29   & $-$134.3 & 0.9 & 50.3  & --           & 3    \\
            {HD 168443}b & $-0.047\pm0.552$  & 58.1124 & $4e^{-4}$ & 0.529 & 0.001 & 475.13 & 0.9  & 5626.20 & 0.02 & 172.9    & 0.1 & 7.7   & --           & 4    \\
            {HD 168443}c & $-0.047\pm0.552$  & 1749.83 & 0.57      & 0.211 & 0.002 & 297.7  & 0.6  & 5521.3  & 2.2  & 64.9     & 0.5 & 17.1  & --           & 4    \\
            {HD 202206}B & 14.721            & 256.33  & 0.02      & 0.432 & 0.001 & 567    & 1    & 2176.14 & 0.12 & 161.9    & 0.2 & 17.4  & $93.2\pm7.3$ & 5, 6 \\
            {HD 202206}c & 14.721            & 1260    & 11        & 0.22  & 0.03  & 41     & 1    & 3103    & 452  & 280      & 4   & 2.3   & $17.9\pm2.9$ & 5, 6 \\
            {HD 211847}  & 6.689\tnote{a}    & 7929.4  & 2500      & 0.685 & 0.068 & 291.4  & 12.2 & 12030.1 & 2500 & 159.2    & 2.0 & 19.2  & 155          & 3, 7 \\
            {HD 30501}   & $23.710\pm0.028$  & 2073.6  & 3.0       & 0.741 & 0.004 & 1703.1 & 26.0 & 3851.5  & 3.0  & 70.4     & 0.7 & 62.3  & 89.6         & 3    \\
            \bottomrule
        \end{tabular} \label{tab:orbitparams}
        \begin{tablenotes}
            \item [a]{fixed}
            \item[]References: (1)~\citet{crepp_trends_2016}; (2)~\citet{udry_coralie_2002}; (3)~\citet{sahlmann_search_2011};   (4)~\citet{pilyavsky_search_2011}; (5)~\citet{correia_coralie_2005};  (6)~\citet{benedict_hd_2017}; (7)~\citet{moutou_eccentricity_2017}
        \end{tablenotes}
    \end{threeparttable}
\end{table*}


%--------------------------------------------------------------------
\section{Data and data reduction}
\label{sec:data}
In this section we explain the observations and the data reduction process used. We also discuss the models used to correct for atmospheric absorption.


\subsection{CRIRES data}
\label{subsec:CRIRES} 
Observations were performed with the CRIRES instrument~\citep{kaeufl_crires_2004} configured to observe a narrow wavelength domain of the \textit{K}-band between 2120--2160~nm using the {Ks} and the {Hx5e-2} filters. The slit width of \(0.4\sec \) resulted in an instrumental resolving power of \(\rm R=50\,000\), with no adaptive optics to ensure that the slit was entirely covered by each target. This prevents strong slit illumination variations that could change the shape of spectral lines.

The observations were performed in service mode during period 89 with run ID.~089.C-0977(A) between April and August 2012. An observation is composed of 8 individual spectra each with an integration time of 180 seconds, observed in a ABBAABBA nod cycle pattern to obtain a high (>180) signal-to-noise when combined. 

{\rd The list of observations obtained with CRIRES are provided in Table~\ref{tab:observations}. The number of artefacts found in the 32 individual detector nod spectra for each observation are listed as artefacts, see Section~\ref{subsubsec:reduction} for more details. The SNR is calculated with the formula \(\textrm{SNR} = \frac{\mu}{\sigma}\) where $\mu$ and $\sigma$ are the mean and standard deviation in the continuum of detector 2 between 2130 and 2134 (see Fig.~\ref{fig:spectral_example}).
The estimated RV values for the host and companion at the time of each observation are calculated using Equation~\ref{eq:rv_equation} using the best known orbital parameters and the companion masses from Tables~\ref{tab:starparams} and \ref{tab:orbitparams}. For hosts with multiple companions the RV value is for the largest companion only, i.e. {HD 202206}B and {HD 168443}c. The RV difference between the host and the companion \(rv_2\) = (\(RV_2 - RV_1\)) is parameter used in the binary model from Sect.~\ref{subsubsec:binary-model}. }

%!TEX root = ../nir_companions.tex

% Table of observations
\begin{table*}
    \small
            \centering  
           \begin{threeparttable}[b]
     
            \caption{Details about the each CRIRES observation. {\rd{} The time, settings, number of artefacts removed, the SNR obtained and the predicted orbital state of each system are provided.}}
            %\begin{tabular}{l c c c c cl cl r@{.}l r@{.}l r@{.}l}
            \begin{tabular}{l c c c c c c r@{.}l r@{.}l r@{.}l}
                \toprule
                Object & Obs. \# & Start date  & Filter & Airmass  & Artefacts & SNR & \multicolumn{2}{c}{\(RV_1\)} & \multicolumn{2}{c}{\(RV_2\)} & \multicolumn{2}{c}{\(rv_2\)}  \\  % & \(Date \)
                &   & (yyyy-mm-dd hh:mm:ss)  &  & (at start) & {\rd / 32} & & \multicolumn{2}{c}{kms\(^{-1}\)} & \multicolumn{2}{c}{kms\(^{-1}\)} & \multicolumn{2}{c}{kms\(^{-1}\)}\\ % data ref    % & (JD\(^{\star} \))
                \midrule
                {HD 4747}   & 1 & 2012-07-06 07:36:06 & Ks     	      & 1.25  	  & 7 & 340 & $-$0    & 219 & $-$0  & 154 & 0&065 \\ %-1      & 2456114.81674
                {HD 162020} & 1 & 2012-07-04 06:23:22 & Ks     		& 1.30 		& 2 & 127 & $-$28  & 760 & 50 & 785\tnote{a}  & 79&545\tnote{a} \\ %-1      & 2456112.76624
                {HD 162020} & 2 & 2012-07-04 06:57:48 & Ks     		& 1.44  	& 2 & 128 & $-$28  & 717 & 48 & 440\tnote{a} & 77&157\tnote{a} \\ %-2      & 2456112.79015
                {HD 167665} & 1 & 2012-07-28 05:00:53 & Hx5e-2 	& 1.24 		& 7 & 371 & 7         & 581 & 18 & 024\tnote{a} & 10&443\tnote{a} \\ %-1a     & 2456136.70895
                {HD 167665} & 2 & 2012-07-28 05:37:27 & Hx5e-2 	& 1.39  	& 4 & 374 & 7         & 581 & 18 & 025\tnote{a}  & 10&444\tnote{a} \\ %-1b     & 2456136.73434
                {HD 167665} & 3 & 2012-08-05 02:54:03 & Hx5e-2 	& 1.04  	& 4 & 358 & 7         & 575 & 18 & 163\tnote{a} & 10&588\tnote{a} \\ %-2      & 2456144.62087
                {HD 168443} & 1 & 2012-08-05 04:29:32 & Ks     		& 1.31 		& 2& 192 & $-$0   & 121 & 50 & 932\tnote{a,b}  & 51&053\tnote{a,b} \\ %-1      & 2456144.68718
                {HD 168443} & 2 & 2012-08-05 04:58:50 & Ks     		& 1.47 		& 4 & 190 & $-$0   & 121 & 51 & 189 \tnote{a,b} & 51&310\tnote{a,b} \\ %-2      & 2456144.70753
                {HD 202206} & 1 & 2012-07-12 06:54:44 & Ks     		& 1.01 		& 3& 189 & 14      & 843 & 12 & 992\tnote{b}  & -1&851 \\ %-1      & 2456120.78801
                {HD 202206} & 2 & 2012-07-13 05:41:40 & J       	  & 1.01 	  & 3 & 209 & 14      & 837 & 13 & 065\tnote{b}  & -1&772 \\ %-2      & 2456121.73727
                {HD 202206} & 3 & 2012-07-11 08:29:55 & Ks     		& 1.15		& 4& 180 & 14      & 849 & 12 & 920\tnote{b}  & -1&929 \\ %-3      & 2456119.85411
                {HD 211847} & 1 & 2012-07-06 07:02:57 & Ks     		& 1.07 		& 4& 272 & 6        & 613 & 7   & 171 & 0& 558\\ %-1      & 2456114.79372
                {HD 211847} & 2 & 2012-07-13 06:54:37 & Ks     		& 1.05 		& 5& 283 & 6        & 614 & 7   & 167 & 0&553 \\ %-2      & 2456121.78793
                {HD 30501}  & 1 & 2012-04-07 00:08:29 & Hx5e-2 	 & 1.60 	 & 3& 217 & 22      &  372 & 36 & 377 & 14&005 \\ %-1      & 2456024.50590
                {HD 30501}  & 2 & 2012-08-01 09:17:30 & Hx5e-2    & 1.42     & 10& 212 & 22      & 505 & 35  & 120 & 12&615 \\ %-2a     & 2456140.88716
                {HD 30501}  & 3 & 2012-08-02 08:47:30 & Hx5e-2 	 & 1.53 	 & 8& 237 & 22      & 507 &  35 & 102 & 12&595 \\ %-3      & 2456141.86633
                {HD 30501}  & 4 & 2012-08-06 09:42:07 & Ks     		 & 1.28 	 & 7& 235& 22      & 514 & 35 & 031 & 12&517 \\ %-2b     & 2456145.90426
                \bottomrule
                & & & & 
        \end{tabular}
        \label{tab:observations}
        \begin{tablenotes}
          \item  [a]{Maximum RV given \(\textrm{M}_2\sin{i}\) only.}
          \item  [b]{Largest mass companion only.}
        \end{tablenotes}
    \end{threeparttable}

\end{table*}


\subsubsection{Data reduction}
\label{subsubsec:reduction}
\begin{figure}
    \includegraphics[width=\hsize]{images/Bad_pixel_replacement.pdf} % chktex 8
    \caption{Example of an artefact in the optimally extracted spectra. The top panel contains the 8 normalized nod spectra after optimal extraction, while the middle panel shows the rectangular extraction for the exact same spectra. A vertical offset is included between each spectra. A single large spike in the seventh nod (pink) near pixel 230 creates a wide and noticeable artefact in the optimal extraction. The bottom panel shows the difference between a combined spectrum using the optimal nods only and a combined spectrum in which the seventh nod is replaced with its rectangular counterpart. The nod spectra are in observation order from top to bottom.}
    \label{fig:nod_artefacts}
\end{figure}

The observations were reduced using a custom reduction pipeline~\citep{figueira_radial_2010}. Written in IRAF's CL\footnote{IRAF is distributed by the National Optical Astronomy Observatories, which are operated by the Association of Universities for Research in Astronomy, {Inc.}, under cooperative agreement with the National Science Foundation.}~\citep{tody_iraf_1993} it provides for automated dark and non-linearity corrections (using the non-linearity coefficients provided by ESO), as well as the flagging and replacement of bad pixels. The images were corrected from sensitivity variations by dividing by a flat-field corrected from the blaze function effect. The nodding pairs were mutually subtracted and the order tracing was fitted by {\rd linear or cubic splines selected for each detector}.

There are two types of extraction commonly used. The \emph{rectangular} extraction performs a rectangular aperture sum in the spatial direction while the \emph{optimal} extraction~\citep{horne_optimal_1986} also includes variance weighting to reduce the impact of the noise and deviant pixels on the total flux measurement. {\rd The extracted nod-cycle spectra are normalized by dividing each by a polynomial fitted to the continuum.}

At this point one would normally combine all the optimally extracted nod spectra together to improve the signal-to.noise. {\rd However, we discovered extended artefacts in the optimally reduced nod spectra. We were not able to identify why these artefacts are large and extended but they occur near the presence of a large cosmic rays or bad pixel spike seen in the rectangular extraction. These artefacts were not observed in previous works using this pipeline in the \textit{H}-band, so there may be a wavelength dependent affect. The variance weighting procedure during the optimal extraction is somehow being heavily affected by these spikes.} An example of this can be seen in Fig.~\ref{fig:nod_artefacts} where a spike in the rectangular extraction corresponds to the large extended features in the optimal extraction. The amplitude of these artefacts created flux deviations in the combined optimally extracted spectra up to \(\sim\)0.5\%. 
Therefore, we took measures to remove these artefacts before combining the nod spectra as we are trying to recover companion spectra with expected flux ratios \(\rm {F_2}/{F_1} < 1\% \). {\rd Parameters of the pipeline were adjusted to try and remove these artefacts, such as the $\sigma$ rejection limits and the aperture width with limited success. In a small number of instances allowing the aperture width to be automatically adjusted removed the artefacts.} 

All nod spectra for each observation were visually inspected together, as shown in  Fig.~\ref{fig:nod_artefacts}, and any spectra containing artefacts were marked. The optimally extracted nods (top panel)  that were identified were replaced with their rectangularly extracted counter-parts (middle panel). An iterative 4-\(\sigma \) rejection algorithm\footnote{Found at \url{https://github.com/jason-neal/nod_combination}} was applied to the replacement rectangular extractions to remove the erroneous pixels that created the artefacts. The \(\sigma\) value for each pixel was calculated as the standard deviation of the nearest 2 pixels on either side of all 8 nod spectra. The rejected pixels were replaced using linear interpolation along the dispersion axis. {\rd Out of the 544 nod spectra from individual detectors, 79 (14\%) contained artefacts and were replaced using this technique.}

For the remainder of the paper we use combined spectra constructed by averaging the 8 nod-cycle spectra together, where some of the optimally extracted spectra have been replaced using the above method. The last panel of Fig.~\ref{fig:nod_artefacts} shows that the difference between the combined optimally extracted spectra and the \emph{combined extraction with replacements}. 

The continuum normalization and nod combining steps were also performed using IRAF while the following post reduction procedures and analysis all utilize \emph{Python}. This pipeline was chosen over the ESO CRIRES pipeline because it seemed relatively simple to use, being semi-automated, and appeared to have less bad pixel/cosmic ray artefacts in the extracted spectra. In hindsight this was not the case, with the removal of the extended artefacts that appeared. 

% One possible explanation for the artefacts present is an instrumental effect, such as instrument glow. This is known to affect CRIRES and included in the data reduction cookbook~\citep{smoker_very_2012}. These artefacts in the \textit{K}-band spectra were not observed in previous works in the \textit{H}-band using this pipeline and as such may have a wavelength dependent affect, as the higher wavelength nIR is more susceptible to thermal instrument glow.


\subsubsection{Wavelength calibration}
\label{subsec:wave_cal}
There is a known issue with the wavelength calibration from the CRIRES Th-Ar calibrations~\citep{kerber_laboratory_2009}, due to the low density of Th-AR lines in the nIR and the alignment with the detectors narrow wavelength range (e.g.\ CRIRES-POP~\citep{nicholls_crirespop_2017}).
Therefore, we use the telluric absorption lines present in each observation as the wavelength reference. Instead of directly using the HITRAN database~\citep{rothman_hitran2012_2013} for the line positions of the telluric spectra (e.g.~\citep{brogi_signature_2012,brogi_carbon_2014,dekok_detection_2013}), we use TAPAS atmospheric transmission models~\citep{bertaux_tapas_2014} obtained for each observation. These in turn use the HITRAN database but include atmospheric profiles and physical measurements to model the telluric absorption strength.

The centroid of each telluric line is obtained by fitting the telluric transmission spectrum, \(T \), as a sum of Gaussian functions (subtracted from the continuum) representing the telluric lines,

\begin{equation}
T(\lambda) = 1 - {\Sigma}_{i}\ G(\lambda, A_{i}, {\mu}_{i}, {\sigma}_{i}),
\end{equation}

where \(G \) is a Gaussian function of the form

\begin{equation}
G(\lambda, A, \mu, \sigma) = {A \textrm{e}}^{{-(\lambda-\mu)}^{2}/2\sigma^{2}}
\end{equation}

and \(A \), \(\mu \), \(\sigma \) are the amplitude, central wavelength, and standard deviation for each line respectively. {\rd Although telluric lines are actually Voigt profiles, they are not fully resolved in the nIR and their shape is dominated by the instrumental profile. The instrumental profile for CRIRES  has been shown to be well represented by a Gaussian \citep{seifahrt_synthesising_2010}.}

The observed spectra contain two different components: stellar and telluric lines overlapped. This overlapping is, in fact, a multiplication of the stellar and telluric spectra. These were fitted with two Gaussian-sum models multiplied together, with the identification of telluric and stellar lines performed {\rd by hand} for each spectra, using the synthetic telluric models as the reference.
\begin{align}
I_{obs}(x) &= I_{tell}(x) \times I_{space}(x) \nonumber \\
I_{obs}(x) &= \Big(1 - {\Sigma}_{j}\ G(x, A_{j}, {\mu}_{j}, {\sigma}_{j})\Big) \times \Big(1 - {\Sigma}_{k} G(x, A_{k}, {\mu}_{k}, {\sigma}_{k})\Big), \label{eqn:obs}
\end{align}

where \(x \) is the pixel coordinate of the extracted spectra.

The wavelength solution was obtained by fitting a second order polynomial, shown to be sufficient for higher precision RV studies~\citep[e.g.][]{bean_groundbased_2010, figueira_radial_2010, seifahrt_synthesising_2010}, to the centroid values \(\{\mu(x), \mu(\lambda)\} \) obtained from the telluric component of the observed spectra and telluric model respectively. 

Much like issues with Th-Ar calibrations this method only works well when there is sufficient coverage of telluric lines on the detector. For the wavelength setting of these observations, the spectra from the second detector (top right panel of Fig.~\ref{fig:detector4allspectra}) only has two large telluric lines present with several small lines, with relative depths smaller than 1\%, which are difficult to identify. This deteriorated the calibration stability for the second detector. With the lack of telluric contamination and stellar lines on the second detector it may have been ideal for the detection of a faint secondary spectra; unfortunately the wavelength calibration quality varies in an inverse way.

We note that there are many variations on this wavelength-calibration technique including those integrated within programs such as \emph{TelFit}~\citet{gullikson_correcting_2014}, and ESOs \emph{Molecfit}~\citet{smette_molecfit_2015}, that perform telluric correction and re-calibrate the wavelength axis themselves. {\rd Including concurrent fitting of a stellar spectral model, adjusted for RV, along with the telluric model could help to improve the wavelength calibration preformed here. For example \citet{piskorz_evidence_2016} use only a telluric line model at other nIR wavelengths but needed to include a stellar model specifically around 2\,$\mu$m where the telluric lines are weaker.}


\begin{figure*}
    \includegraphics[width=0.8\hsize]{images/Spectra_examples.pdf}
    \caption{Extracted, normalized and wavelength calibrated spectra for a single observation of each target. The target name is given above each spectrum along with the observation number. Each panel is the spectra from a single detector 1--4 in order of increasing wavelength. The black dashed lines indicate the unique telluric spectrum used for wavelength calibration and telluric correction for each observation.}
    \label{fig:detector4allspectra}
\end{figure*}


\subsubsection{Telluric correction}
\label{subsec:telluric_correction}
Ground based observations require the removal of the absorption lines introduced by Earth's atmosphere. These observations were first taken in an atmospheric window of the \textit{K}-band in order to reduce the absorption introduced by the atmosphere~\citep{barnes_hd_2008}. To correct for the remaining telluric line contamination the spectra were divided by the TAPAS\citep{bertaux_tapas_2014} atmospheric transmission models for each observation. Synthetic telluric models were used to avoid the observing overhead necessary to perform telluric standard star exposures~\citep{vacca_method_2003}, and they have been demonstrated to be superior in the quality of the correction relative to the telluric standard approach~\citep[e.g.][]{cotton_atmospheric_2014}.

Before the correction, the depth of the telluric lines were re-scaled to match the airmass of the observation using the relation \(\rm T = T^{\beta} \), where \(\rm T\) is the telluric spectrum and \(\beta \) is the airmass ratio between the observation and model. This changed the depth of most absorption lines to match the observations, but does not correctly scale the deeper \(\rm H_{2}O \) lines. The scaled telluric model is interpolated to the wavelengths of the observed spectrum and then used to correct the observed spectra through division, leaving behind a telluric corrected spectra. An example of a telluric corrected spectra is shown in the middle panel of Fig.~\ref{fig:spectral_example}, with the light blue shading indicating where the deeper telluric lines were.

We attempted the technique suggested by~\citet{bertaux_tapas_2014} to address the poor \(\rm H_{2}O \) airmass scaling, to fit a scaling factor to the \(\rm {H}_{2}O \) absorption lines before convolution to the instrument resolution. This was achieved by first dividing the spectrum by a telluric model with only non-\(\rm H_{2}O \) constituents, convolved to the observed resolution, and scaled by the airmass to remove the non-\(\rm H_{2}O \) lines. Then a model with only \(\rm H_{2}O \) lines at full resolution was scaled by a factor \(\textrm{T}^{x} \), convolved to \(\rm R=50\,000 \) and compared to the observed spectra. The factor \(x \) was fitted to find the best scaling factor for the \(\rm H_{2}O \) lines.

We found that for a few spectra in our sample this method corrected the deeper telluric lines well, but in many cases we found that the fitted scaling factor was affected by the presence of blended stellar lines (attempting to fit those also). It was also strongly influenced by the deepest \(H_{2}O\) telluric lines present. We find that the telluric correction of the deep \(\rm H_{2}O \) lines could be improved with this technique, but, at the cost of worsening the correction of the many smaller \(\rm H_{2}O \) lines. Since the smaller \(\rm H_{2}O \) lines covered more of the spectrum in this region than the larger lines we chose not to continue with this separate \(\rm H_{2}O \) scaling. One possible solution for this would be to perform a piece-wise telluric correction, performing this step only for the deeper \( \rm H_{2}O\) lines, or by using one of the other tools that fits the telluric model to the observations. This technique could also benefit from a larger wavelength span that would enable blended lines to be ignored while having sufficient deep \(\rm H_{2}O\) lines to fit the scaling factor correctly. This small experiment shows that a simple scaling is not enough to correct for the absorption in an effective way, for this case.

{\rd  The recent work of \citet{piskorz_evidence_2016} use a PCA technique to correct for the telluric spectra by applying it to several (number not specified) AB nodding pairs over their 60-180 minute integration time. We are uncertain if this technique would work as effectively on our observations due to our shorter integration time (24 minutes) would have less telluric variation present across the 8 nod spectra . A recent comparison of three telluric correction methods, \emph{Molecfit} and {TelFit} and {TAPAS} to the standard star method by \cite[][in prep.]{ulmer-moll_telluric_2018} found that all three synthetic models preform better at correcting for atmospheric H$_2$O compared to the standard star method with \emph{Molecfit}, being a more complete tool, preforming slightly better over TAPAS.}

After the telluric correction is performed, the spectra are corrected for Earth's barycentric RV using the \emph{helcor} PyAstronomy\footnote{https://pyastronomy.readthedocs.io} function ported from the REDUCE IDL package (See~\citet[][]{piskunov_new_2002}).


\subsection{Tapas models}
\label{subsec:tapas_models}
We used telluric line models to wavelength calibrate the reduced spectra in Sect.~\ref{subsec:wave_cal} and to correct for the atmospheric absorption in Sect.~\ref{subsec:telluric_correction}. We utilized the TAPAS (Transmissions of the AtmosPhere for AStronomical data) web-service\footnote{\url{http://www.pole-ether.fr/tapas/}}~\citep{bertaux_tapas_2014} to obtain atmospheric transmission models for each observation. TAPAS uses the standard line-by-line radiation transfer model code LBLRTM~\citep{clough_linebyline_1995} along with the 2008 HITRAN spectroscopic database~\citep{rothman_hitran_2009} and Arletty atmospheric profiles derived using meteorological measurements from the ETHER data center\footnote{\url{http://www.pole-ether.fr}}, which has a 6 hour resolution in atmospheric profiles.
We use the mid-observation time to retrieve transmission models for each observation, with the Arletty atmospheric profiles\footnote{Nearest of the 6 hourly profiles} and vacuum wavelengths. The telluric models were retrieved without any barycentric correction to keep the telluric lines at a RV of zero with respect to the instrument. We obtained one model with all provided species present, convolved to a resolution of \(\rm R=50\,000 \), and another two models without an instrumental profile convolution applied. For these two extra models, one contained only the transmission spectra of \(\rm H_{2}O \) while the other was for all other constituents except \(\rm H_{2}O \). This was to explore a known issue~\citep{bertaux_tapas_2014} with the depth of \(\rm H_{2}O \) absorption lines in Sect.~\ref{subsec:telluric_correction}.


\subsection{Wavelength masking}
We apply several wavelength masks to remove wavelength regions from which we cannot extract information.
Firstly, regions near the edges of each detector where the wavelength solution is extrapolated outside of the calibrating telluric lines are removed, reducing the effective size of each detector by about \(10\%\) or \(\sim\)100~pixels. 

Secondly, we mask out any remaining artefacts present in the spectra and the centers of deep telluric lines where telluric correction was not corrected properly, sometimes resulting in ``emission-like'' peaks in the corrected spectrum. These factors combined result in masking out around a further 10\% of the observed spectra. 

In Sect.~\ref{sec:results} we also apply a further wavelength restriction to mask out regions where there is a large mismatch between the observed spectrum and the closest synthetic spectra to the host. This significantly restricts the wavelength span utilized for that purpose to around only 43\% and the masked regions are visible in Fig.~\ref{fig:visualinspection-hd2118471}. 


\section{Differential subtraction}
\label{sec:spec_diff}

The observations were gathered having in mind the application of a differential subtraction method~\citep[e.g.][]{ferluga_separating_1997, kostogryz_spectral_2013} to detect the spectra of the faint BD companions. In short, we Doppler shift each observation to the rest frame of the host star and then mutually subtract the spectra from pairs of observations to cancel out the spectra of the brighter host star. The residuals from this method should contain two copies of the faint companion, subtracted from each other with a radial velocity offset \(\Delta RV\) between them. This RV offset, if detectable, would allow the mass ratio to be determined and hence the companion mass. 

Due to the poorly separated observation times relative to the long orbital periods, this method was revealed to be inappropriate for these observations as the RV separation of the companion spectra between observations (\(\le 2.3\)\kmps{}) is significantly smaller than the FWHM (full width half maximum) of individual spectral lines (\(\sim\)6\kmps at \(\rm R=50\,000\)).  The small separation of the companion causes the lines of the companion to also mutually cancel, severely reducing the residual signal to well below the available noise level. The requirement of well separated RVs for the companion spectra was clearly stated in the original proposal but was not satisfied when the observations were obtained\footnote{see Sect.~\ref{subsubsec:differential-schedualing} for more details}. {\rd The very large orbital periods of some of the targets would not produce a sufficient RV signal during one semester. This was a possible oversight during the proposal stage.} The largest estimated companion \(\Delta RV\) separation between the observations of each target is provided in the seventh column of Table~\ref{tab:estimatedparameters}. Radial velocity constraints are also valid for other studies such as the detection of reflected light from exoplanets~\cite{martins_evidence_2015}. 

Despite these problems, we still tried to apply the method to our data. Given the negative result, however, we decided to move this section to Appendix~\ref{appendix:A1}, where we describe the method and some exploratory results that may be useful for future studies.

%%%%%%%%%%%%%%%%%%%%%%%%%%%%%%%%%%%%%%%%%%%%%%%%%%%%%%%%%%%%%
%%%%%%%%%%%%%%%%%%%%%%%%%%%%%%%%%%%%%%%%%%%%%%%%%%%%%%%%%%%%%

\section{Binary synthetic spectral recovery}
\label{subsec:companion_recovery}
Since the differential method is ineffective for our dataset due to the RV separation between observations, we explored a second approach to detect the presence of the faint companion spectra. This second approach compares the observed spectra to combinations of synthetic spectral models using \(\chi^{2}\) methods which have been extensively used in the literature~\citep[e.g.][]{astudillo-defru_harps_2015, passegger_fundamental_2016, zechmeister_spectrum_2018,nemravova_xtauri_2016}. {\rd The temperature of synthetic spectra fitted to  the companion will provide some indication of the companions spectral type, but will not produce a direct mass constraint that was the original aim of this work.}

\subsection{Synthetic PHOENIX-ACES models}
\label{subsec:spec_models}
We use the PHOENIX-ACES~\citep{husser_new_2013} synthetic spectra library as our reference for the spectral comparison. It uses the most recent version (16) of the PHOENIX code and is suitable for the spectra of cool stars. The full parameter grid space of the PHOENIX-ACES spectra is given in Table~\ref{tab:phoenix} although we only explore models constrained by the targets explored here.

%!TEX root = ../nir_companions.tex

\begin{table}
    \centering 
    \caption{Full parameter space of the PHOENIX-ACES spectral grid.}
    \begin{tabular}{lr@{ -- }lc}    % Seperate columns with --
        \toprule
        & \multicolumn{2}{c}{Range}       & Step size\\
        %\midrule
        \midrule
        \ \(T_{\textrm{eff}}\) [K] &  2300 & 7000  & 100 \\
        &  7000 & 12000 & 200 \\ 
        \  logg     &  0.0 & +6.0   & 0.5 \\ 
        \ [Fe/H]   &  -4.0 & $-$2.0  & 1.0 \\    % Strange spacing of [ ] in table so added \ to all rows
        &  -2.0 & +1.0  & 0.5 \\  
        \  \(\alpha\)/Fe &  -0.2 & +1.2  & 0.2 \\
        \bottomrule
    \end{tabular}
    \label{tab:phoenix}
\end{table}

The spectral model libraries were accessed using the useful ``grid tools'' interface provided in the \emph{Starfish}\footnote{\url{https://github.com/iancze/Starfish}} Python package~\citep{czekala_constructing_2015}, which made it efficient to load in the spectra when needed.

We multiply the synthetic spectra by the wavelength to convert it into photon counts, ignoring multiplicative constants, as done in~\citet{figueira_radial_2016}\footnote{Synthetic models provide the spectral energy distribution (\(\rm erg\,s^{-1}\,cm^{2}\,cm^{-1}\)).}. The spectra were convolved with a Gaussian kernel to match the resolution of the observations (\(\rm R=50\,000\)). Due to the distributive property of convolution it is efficient to apply it once to each spectra first, before the spectral pairs are combined.

The PHOENIX-ACES models include dust in equilibrium with the gas phase while ignoring dust opacity and does not include any mixing/settling which is important for cooler BD atmospheres. They set a minimum library \(T_{\textrm{eff}}=2300\)~K to avoid the temperatures at which the modelling of clouds is necessary. This unfortunately limits the use of this library for this technique to the largest mass companions in our sample. For example a \(T_{\textrm{eff}}=2300\)~K corresponds to a BD with \(\rm M\sim84~M_{Jup}\) at 5 Gyr from the~\citet{baraffe_evolutionary_2003} evolutionary models. 

There are other models that extend below 2300~K such as the {BT-Settl} models\citep{allard_btsettl_2013,baraffe_new_2015}. These are discussed in Sect.~\ref{subsubsec:BT-Settl}.


\subsection{\texorpdfstring{\(\chi^{2}\)}\ \ method}
\label{subsec:chi2}
The well known \(\chi^{2} \) technique measures the weighted sum of the squared deviation between the observation (\({O}_{i}\)) and the computed models (\(C_{i}\)), with the minimum \(\chi^2\) value representing the best-fit parameters.
\[\chi^{2} = {\Sigma}_i { (O_{i} - C_{i})}^2 / {\sigma}_{i}, \] where \({\sigma}_{i}\) is the error on each measurement. We estimate the \(\sigma\) of each spectrum using the \(\beta\sigma\) method~\citep{czesla_posteriori_2018}, using the MAD (median absolute deviation about the median) robust estimator. {\rd This method estimates the spectral noise using numerical derivatives of the spectra. We followed the procedure outlined in \citet{czesla_posteriori_2018}, analysing the results from successive parameter combinations to settle on an order of approximation (derivative level) of 5, and a jump parameter (pixels skipped to avoid correlations) of 2.} We apply the same \(\sigma\) value to all points \({\sigma}_{i} = \sigma\). The \(\beta\sigma\) method provided \(\sigma\) estimates for the target spectra which correspond inversely to signal-to-noise ratios between 100--500, {\rd similar to the values given in Table~\ref{tab:observations} calculated from the continuum of detector 2.}

The computed models are described in Sect.~\ref{models} and result in a multidimensional grid of \(\chi^2\) values for each combination of parameters, namely the spectral temperature, host RV, companion RV) for each detector, observation and target.

We obtain the global minimum of the multidimensional \(\chi^{2}\)-space to represent the best fit to the observed spectra. We sum the multidimensional \(\chi^{2} \) across multiple detectors and determine a global minimum \(\chi^{2} \) for the whole observation \(\chi^{2}_{obs} = \Sigma^{N}_{n=1} \chi^{2}_n \), where \(N\) is the number of detectors used. We do not, however, combine the \(\chi^{2} \) values across the separate observations as the RV parameters of the host and companion will vary between each observation. However, since the current observations are insufficiently separated, it may be possible to combine the separate observations; but in general this would not be the case, so was not performed.

The inverse survival function of the \(\chi^2\) distribution is used to determine the confidence levels on the minimum \(\chi^2\) parameters. The inverse survival function returns a \(\Delta\chi^2\) value from the minimum \(\chi^2\) value for a given sigma level and degree of freedom\footnote{In \emph{Python} with the \emph{scipy} package this is a single line \texttt{scipy.stats.chi2(dof).isf(1-p)}, where \(p = 0.68\) for 1-\(\sigma\), and dof is the degree of freedom.}. 
For example, the \(\Delta \chi^2\) for a single degree of freedom required for the 1-, 2-, and 3-\(\sigma\) levels is 1, 4, and 9 respectively~\cite{bevington_data_2003}. This method assumes that the measured flux is observed with a SNR sufficiently high so that the noise on the spectrum is approximately Gaussian, and the \(\chi^2\) method appropriate.

For a given observation, the \(\chi^{2}_{red}\) is computed by \(\chi^2_{red} = \chi^2 / \nu \) where \(\nu = n - m\), the number of observed pixels, \(n\), minus the number of parameters of interest, \(m\)\footnote{\(m=2\) or 4 in the examples explored below}, and is performed after the summation over the detectors.


\subsection{Computed model spectra}
\label{models}
In this section we detail how we transform the synthetic PHOENIX-ACES spectra into the computed models (\(C_i\)) to fit to the observations. The spectra have already been converted to the correct unit and resolution.

These synthetic spectra are used individually for the single component model and combined together into a binary model. The results of these models are interpolated to the wavelength grid of the observed spectra and the \(\chi^{2} \) calculated by comparing the model and observation at each point.


\subsubsection{Single component model}
\label{subsubsec:single-model}
The single model \(C^{1}_{i}\) comprises of a single synthetic spectrum, \(J\), (with model parameters \(T_{\textrm{eff}}\), logg, [Fe/H], ([\(\alpha\)/Fe]) and is Doppler shifted by a RV value \({rv}_1\). 

\begin{equation}
\rm C^{1}_{i}(\lambda) = J(\lambda_0(1-\frac{{rv}_1}{c}))
\end{equation}

where \(\lambda\) is the shifted wavelength, \(\lambda_0\), the model rest wavelength and, \(c\), the speed of light in a vacuum. The model's flux is then continuum normalized to unity to match the observed spectra, and interpolated to the wavelength grid of the observation.

This single component model analysis is similar to the~\citet{passegger_fundamental_2016} \(\chi^2\) fitting. We apply the same re-normalization (see Sect.~\ref{subsec:renorm}) to account for slight differences in the continuum level and possible linear trends between the normalized observation and model. We do not, however, apply any dynamical masking to sensitive lines to make the the \(\chi^2\) minima more distinct or linearly interpolate the stellar parameters between the grid models to obtain high precision stellar parameters. This is because we are not trying to derive precise stellar parameters but to detect the spectra of the companions. We instead include radial velocity components to the \(\chi^2\) fitting, which is not included in~\citet{passegger_fundamental_2016}. 


\subsubsection{Binary model}
\label{subsubsec:binary-model}
In the binary situation we consider the superposition of two synthetic spectral components, one each for the host and companion respectively. Both spectra are Doppler shifted by \({rv}_1\) which represents the RV motion of the host star, while the companion spectra is also Doppler shifted by a second RV, \({rv}_2 \), representing the RV offset between the host and companion. This choice is arbitrary, but in this way the mean motion of the system relative to Earth is captured only in \({rv}_1\). The two spectra are scaled by their squared radius (see Sect.~\ref{subsection-radius}) then added together, thus fixing the relative amplitude of the two components. 
Given two spectral components \(J_{1} \) and \(J_{2} \) with radii \(R_1, R_2\) this equates to
\begin{align}
\rm C^{2}_{i}(\lambda) = &  J_{1}(\lambda_{0}(1 - \frac{rv_{1}}{c}))\times R_{1}^2 +\nonumber \\
& J_{2}(\lambda_{0}(1-\frac{rv_{1}}{c})(1-\frac{rv_{2}}{c}))\times R_{2}^2
\end{align}


The combined spectra is continuum normalized by dividing by an exponential fitted to the continuum of the combined spectrum, as we assume we are in the Rayleigh-Jeans regime. This assumption here is wavelength dependent and other appropriate continuum normalization techniques are also valid. In the case of a BD companion around an FGK star investigated here, the continuum is dominated by the contribution from the host star as it contributes the majority of the spectrum with flux ratios around 2110--2160~nm below \(\sim\)1\%.

We combine the models in this way to represent the correct absolute flux ratio of the components. The median flux ratio between the two components is calculated for the wavelength range used here as an indication of the flux ratio level.

The binary model should provide meaningful information about the companion parameters (e.g.\ \(T_{\textrm{eff}}\)) and a estimate of the flux ratio of the system. These can be combined with the~\citet{baraffe_evolutionary_2003} models to constrain the mass of the companion. However, we need to be careful with this model as the inclusion of extra spectral components and associated parameters could also provide a better fit to observations which do not have a companion, by fitting components of the noise.\\

{\rd The full list of grid parameters for the binary model are \(T_{\textrm{eff}}1\),  \(\rm logg_1\), [Fe/H]$_1$, [\(\alpha\)/Fe]$_1$, ${rv}_1$, \(T_{\textrm{eff}}2\), \(\rm logg_2\), [Fe/H]$_2$, [\(\alpha\)/Fe]$_2$, ${rv}_2$ where the subscripts 1 and 2 indicate the host and companion models respectively.}
\subsubsection{Effective radius}
\label{subsection-radius}

To combine the two synthetic spectra with the correct observed flux ratio we need to integrate the emitted flux over the effective surface area of each emitting body respectively. Ignoring the common multiplicative constants that will disappear with normalization, we scale the two synthetic spectra individually by the square of their respective radii, \(R_1\) and \(R_2\). 

In this work we use the effective radius (PHXREFF keyword) of each component from the PHOENIX model headers. This is the radius used in modelling of the stellar atmospheres. We use this radii as it is directly tied to each model spectrum, and already available. The ratio of the radii from the two synthetic spectra in the binary models are provided in Table~\ref{tab:example_params}.

We are aware that using these radii radii has its limitations, since as stated previously, there is a degeneracy in BD mass, age, and luminosity of the companion, and in particular a combination of radius-mass and radius-age relationships~\citep{sorahana_radii_2013}. Using the PHOENIX-ACES model effective radius does not allow for any independent age constraints to be incorporated, or allow for any variability in the radii to account for uncertainties.

The targets analysed here do not have transits, but if the radius ratio can be independently determined from the photometric transit method~\citep{deeg_photometric_1998} then this could be used to constrain the radius ratio used when combing the binary model spectra instead.


\subsection{Re-normalization}
\label{subsec:renorm}
Slight trends in the continuum level between the observed spectra and computed models were removed using the re-normalization following~\citep{passegger_fundamental_2016}:
\begin{equation}
F^{obs}_{re-norm} = F^{obs} \cdot \frac{\textrm{continuum fit}_{model}}{\textrm{continuum fit}_{observations}}.
\end{equation}
The polynomial continuum fits to the normalized observations and models are used to re-normalize the observed spectrum to the continuum of the models. For detectors 1--3 a polynomial of first degree was used, while for detector 4 a quadratic is needed to fit the edge of a strong Hydrogen line (Brackett-\(\gamma\)) at 2166~nm, which lies just off of detector 4. This broad line is only observed in the synthetic spectra and not in the reduced observations. It is assumed that this was normalized out during the reduction process.

For each model we further allow the continuum level to be varied by \(\pm 0.05\) as a free parameter taking the model with the smallest \(\chi^2 \) value.


\subsection{Reducing parameters}
\label{subsec:reduce-params}
The high dimensionality of the binary model makes it computationally challenging and difficult to analyse the \(\chi^2\) space. 
For reference, the multiplicative parameter space is squared when increasing from one spectral component in the single model to a binary model, and therefore becomes computationally expensive. In general the number of possible parameter combinations for \(c\) spectral components each with a grid of \(m\) models increases to \(c^m\). If the full set of PHOENIX-ACES library spectra (66456) is explored with a binary fit then this naively balloons to over 4.4 billion possible combinations. Half of these are not unique as the host and companion components are swapped. This is the worst case scenario and we implement a number of assumptions to vastly reduce the parameter-space enabling faster computation.

Our first assumption is to restrict ourselves to models with an Alpha element abundance ([\(\alpha\)/Fe]) of zero. This is likely a very good approximation as all our targets have solar metallicity and are thus very likely to belong to the thin disk of the Galaxy, where [\(\alpha\)/Fe] values are close to zero (i.e., solar) -- e.g.~\citet{adibekyan_chemical_2012}. Our second is to assume that we can significantly reduce the search space with literature values for the host star. We fix the metallicity of both components to the closest grid to the literature value (usually [Fe/H] =0.00). We also fix the logg of the host star to its literature values given in Table~\ref{tab:starparams}. The uncertainties on the literature measurements for logg (\(\sim\)0.1) and metallicity (\(\sim\)0.05) are both smaller than the grid steps of 0.5 for these parameters.
For the logg of the companion we use the~\citet{baraffe_evolutionary_2003,baraffe_new_2015} evolutionary model value for the given companions \(\textrm{M}_2\)/\(\textrm{M}_2\sin{i}\) and hosts age.

We use the estimated companion temperatures from the Baraffe evolutionary models given the companion \(\rm M_2\) or \(\textrm{M}_2\sin{i}\) and stellar age as a starting point for the companion spectra grid computation and extend the grid in each direction, within the model limits. For example we show the companion temperature grid spanning \(-600\) to \(+400\)~K in Fig.~\ref{fig:Mdwarf_contours} and \(\pm400\)~K in Figs.~\ref{fig:HD211847_simulated_contours} and~\ref{fig:HD211847_result_contours}.

The large numbers stated above also do not include the RV grid for each component. The RV grid is user defined and the number of spectra /models to consider increases when the RV grid step size is decreased (finer RV resolution). We can reduce the RV grid space significantly by tailoring it to the target being examined. For each target, we use the estimated RV values from the observation time and orbital parameters, given in Table~\ref{tab:observations} as a centre starting point for the \({rv}_1\) and \({rv}_2\) values and increment the RV within a few FWHM around those values, or out to the targets \(\rm K_1\) and estimated \(\rm K_2\) values.

An iterative process could be implemented to refine the RV grids, starting at a larger grid with lower RV resolution then performing a higher resolution grid about the minimum \(\chi^2\) RV values. This was only done manually but could have been automated. One could expect that a good starting RV grid step be governed by the spectral resolution, e.g.\ comparable to the FWHM velocity.

For the companions targets with fully resolved orbits the known RV of  the host star, \({rv}_1\) could also have been fixed. We however left this free to see if the known value would be recovered by the fitting. 


\section{Results}
\label{sec:results}

%!TEX root = ../nir_companions.tex

\begin{table*}
 %   \begin{threeparttable}
    \caption{Input and recovered parameters on simulations and an observation when applying a single (\(\rm C^1\)) and binary (\(\rm C^2\)) models. The logg and metallicity were fixed at \(\rm logg_1 = 4.50\), \(\rm logg_2=5.0\) and [Fe/H]=0.0 equally for both components. Gaussian noise was added to both simulations with a SNR of 150.}
    \centering
    \begin{tabular}{c | *3c | *3c | *3c}
        \toprule
        & \multicolumn{3}{c|}{Simulation 1} & \multicolumn{3}{c|}{Simulation 2} & \multicolumn{3}{c}{Observed {HD 211847}} \\
        \midrule
        & Input & \multicolumn{2}{c|}{Recovered} & Input & \multicolumn{2}{c|}{Recovered} & Expected & \multicolumn{2}{c}{Recovered} \\
        & & \(C^1\) & \(C^2\) & & \(C^1\) & \(C^2\) & & \(C^1\)  & \(C^2\) \\
        \midrule
        \(T_{\textrm{eff}_1}\) & 5800 & 5800 & 5800 & 5700 & 5800 & 5700 & \(5715 \pm 24\) & 5900 & 5800\\
        \(T_{\textrm{eff}_2}\) & 4000 & -- & 3800 & 3200 & -- & 3100 & \(\sim3200\) & -- & >3800\tnote{a}\\
        \({rv}_1\) & 0 & 0.1 & 0 & 6.6 & 6.6 & 6.6 & \(6.6 \pm 0.3\) & 7& 7.6 \\
        \({rv}_2\) &  10 & -- & 9.8 & 0.5 & -- &  -1& \(0.5 \pm 2\) & -- &-12.6\\
        \midrule
        \(R_1/R_2\)& 2.57 & -- & 2.71& 3.16 & - & 3.27 & 3.16 & -- & <2.71\tnote{a}\\
        \(\rm F_2/F_1\)& 0.084 & -- & 0.066 & 0.030 & -- & 0.026 & 0.030 & -- & >0.066\tnote{a}\\
        \(\chi^2\)& -- & 4978 & 3792 & -- & 3746 & 3630  & -- & 37688 & 33860\\
        \(\chi^2_{red}\) & -- & 1.62 & 1.24 & -- & 1.22 & 1.18 & -- & 21.3 & 19.2\\
        \bottomrule
    \end{tabular}
    \begin{tablenotes}
       \item [a] {At the arbitrary upper limit for companion temperature grid (3800~K).}
    \end{tablenotes}
%    \end{threeparttable}
    \label{tab:example_params}
\end{table*}

\begin{figure*}
    \centering
    \includegraphics[width=0.8\hsize]{images/Mdwarf_pcolors.pdf}
    
    \caption{\(\chi^2\) results for companion recovery of a simulated binary observation of a Sun-like star (\(T_{\textrm{eff}_1}=5800\)~K) with an M-dwarf companion (\(T_{\textrm{eff}_2}=4000\)~K). The top right plot shows the application of a single component model (\(C^1\)) while the other three are using a binary model (\(C^2\)). Both left hand panels show the distribution of host temperature and host RV. The top right panel shows the distribution for host and companion temperature, and the bottom right the companion temperature and radial velocity.
        The red circle and yellow star indicate the location of the simulation input and recovered parameters respectively.
        The white line shows a 3-\(\sigma\) confidence level about the minimum \(\chi^2\) solution grid point. Each box is centred on the parameter values and shows the grid resolution.}
    \label{fig:Mdwarf_contours}
\end{figure*}


\begin{figure*}
    \centering
    \includegraphics[width=0.8\hsize]{images/HD211847_example_pcolors.pdf}
    \caption{Similar to Fig.~\ref{fig:Mdwarf_contours}, \(\chi^2\) results for companion recovery of a simulated binary observation similar to {HD 211847}, (\(T_{\textrm{eff}_1} = 5800\)~K, \(T_{\textrm{eff}_2}=3200\)~K). The top right plot shows the application of a single component model (\(C^1\)) while the other three are using a binary model (\(C^2\)). Both left hand panels show the distribution of host temperature and host RV. The top right panel shows the distribution for host and companion temperature, and the bottom right the companion temperature and radial velocity.
        The red circle and yellow star indicate the location of the simulation input and recovered parameters respectively.
        The white line shows a 3-\(\sigma\) confidence level about the minimum \(\chi^2\) solution grid point. Each box is centred on the parameter values and shows the grid resolution.}
    \label{fig:HD211847_simulated_contours}
\end{figure*}
Here we show the results from applying the companion recovery model to simulated observations and to an observation.


\subsection{Simulated binaries}
\label{subsec:simulated_binaries}
To test the companion recovery method we create simulated binary observations using PHOENIX-ACES spectra. White noise was added with a standard deviation \(\rm \sigma = 1/SNR\), for a given signal-to-noise (SNR) level. We then applied the grid-matching recovery technique detailed above and compared the resulting parameters to the inputs. 

The results of two example binary simulations are displayed in Figs.~\ref{fig:Mdwarf_contours} and ~\ref{fig:HD211847_simulated_contours}, both simulated with a SNR of 150. The input and recovered parameters for the binary components are indicated by the red circles and yellow stars respectively, and are given in Table~\ref{tab:example_params}. 
The 3-\(\sigma\) contour is shown in white on the plots to indicate the shape of the confidence level only. The 1-\(\sigma\) contours are not shown here as they are much smaller than the temperature grid step and are not easy to visualize at this scale as they are often smaller than the marker shown at the minimum location. Each coloured rectangle is centred on the grid point, with its shape indicating the resolution of the grid space searched.

The first simulation shown in Fig.~\ref{fig:Mdwarf_contours} is for a Sun-like star with a M-dwarf companion, with a \(T_{\textrm{eff}_2} =4000\)~K. The top-left panel shows the recovered host parameters when the single model is applied to the simulated binary. The top-right and both bottom panels are the parameters recovered when using the binary model. Both left-hand panels display the parameters for the host component to easily compare between models. With both models the host temperature \(T_{\textrm{eff}_1}\) is correctly recovered. The host RV, \({rv}_1\), is 0.1\kmps{} (two grid spaces) different from the simulated value for the single component model and is correctly recovered with the binary model.

The minimum \(\chi^2\) location for the companion temperature is 200~K below the simulated value, and the RV of the companion recovered is 0.2\kmps{} below the input value. The input values for the companion are just outside of the 3-\(\sigma\) contours shown. The flux ratio for the input is 0.08 while the flux ratio recovered is 0.066. 

The second simulation shown in Fig.~\ref{fig:HD211847_simulated_contours} is performed with parameters to mimic the observation of our target with highest flux ratio, {HD 211847}. In this simulation the single component model recovers a host with the correct RV but a temperature 100~K higher than the input value. Again, adding the companion with the binary model recovers the correct host temperature. The companion temperature recovered is 100~K lower than the input temperature and the RV is different by 2\kmps{} which is around one third the FWHM.

In this case with a companion RV offset, \({rv}_2\), near 0\kmps{} the host and companion lines are blended. The same spectral lines from both components are trying to match to the same features of the spectra, making it more difficult to recover the companion parameters. In the bottom right panel there appears to be multiple minima for different \({rv}_2\) and \(T_{\textrm{eff}_2}\) combinations, which we assume is partially due to the small \({rv}_2\).

In both simulations the reduced \(\chi^2_{red}\) for the binary model is closer to 1. This is not surprising as the binary model contains extra parameters. As mentioned above, we need to be careful, as the extra components from the binary may just happen to fit components of the noise when a binary is not present, or in our case has a low flux ratio. 
{\rd We analysis the significance between the two models using the ``Bayesian Information Criterion''(BIC)~\citep{schwarz_estimating_1978}; }
\begin{equation}
BIC = n\ln{(m)} - 2\ln{(\hat{L})}.
\end{equation}
{\rd Here $n$ and $m$ are the number of parameters and data points respectively and \(\hat{L}\) is the maximum of the Gaussian likely-hood function }
 \begin{equation}
 \hat{L} = \left(\frac{1}{\sigma \sqrt{2\pi}}\right)^m \exp{\left(-\frac{\chi^2}{2}\right)},
 \end{equation}
 {\rd written in terms of \(\chi^2\) and a fixed $\sigma$ for all data points. The maximum likely-hood of a Gaussian distribution is equivalent  to minimizing the \(\chi^2\). In both simulations \(\Delta BIC >10\) so the preference of the binary model, with the lower BIC value, over the single component model is considered \emph{significant}.}



\subsection{HD211847 observation}
\label{subsection:results-hd211847}
{HD 211847} is the best candidate for detection as it has a \(\rm 155~M_J\) low-mass star companion~\citet{moutou_eccentricity_2017}. The angular separation of the two bodies is 222 mas (or 11.3 au). Even though it is not a BD it has the highest estimated flux ratio in our sample, of 0.03 based on the \citet{baraffe_new_2015} evolution models and the known companion mass (see. Table~\ref{tab:estimatedparameters}). The angular separation of HD211847B is 222 mas with a projected distance of The result of applying \(\chi^2\) fitting to the second observation of {HD 211847} is shown in Fig.~\ref{fig:HD211847_result_contours}.

For this target the metallicity of both components was fixed to 0.0 and the logg for the host was fixed at 4.5. The logg for the companion is fixed to 5.0, based on the~\citet{baraffe_new_2015} evolutionary models for the given companion mass and system age. The orbital solution was used to refine the RV search space of both components. The span RV for the companion was extended until a value inside the RV bounds was found.

Again the top left panel of Fig.~\ref{fig:HD211847_result_contours} shows the recovery with a single component model with the other three for the binary model. The single component model finds a temperature of 5900~K for the host with a \({rv}_1\) of 7\kmps{}. This is 200~K and 0.4\kmps{} different above the expected parameters. The binary model finds a host temperature of 5800~K, which is the second closest model to the literature value, >100~K different. The host RV value recovered with the binary model is 7.6\kmps{}, which is 1\kmps{} higher than expected.  For the single component model there is a barely noticeable secondary minima near this 7.6\kmps{} RV value recovered by the binary model. Again these RV differences are smaller than the FWHM of the lines. The 3-\(\sigma\) contour is small, just visible on the right hand side of the star in the bottom left panel, and hidden behind the markers in the other panels. 


For the companion in the binary model, on the right side of Fig.~\ref{fig:HD211847_result_contours}, the minimum \(\chi^2\) for the companion temperature is at the upper temperature limit of the grid shown. If we extend the grid of companion temperature towards higher temperatures the best fit location continues to increase in temperature, continually hitting the upper limit until it is close to the host temperature, >2000~K above the expected companion temperature. When the companion temperature becomes this high it also affects the recovered parameters for the host star to offset the features of the brighter companion. 

The \(\chi^2_{red}\) values for the single and binary models are 21 and 19 respectively, far from 1, indicating that both models are a poor fit to the observations. {\rd The $\Delta BIC = 3812 >10$ indicating that binary model is still preferred.} We plot the binary model for the best fit solution alongside the observed spectra in Fig.~\ref{fig:visualinspection-hd2118471}. We see that there is a large spectral mismatch between the synthetic models and the observation. Extra wavelength masking was applied to many of the largest mismatched synthetic lines to remove their influence. The grey areas mark regions which have been masked out, either from the centres of deep telluric lines (the thin masks matching spectral gaps), or the more prominent mismatched lines in the synthetic spectrum excluded from the \(\chi^2\) analysis. One clear example of a mismatched line is a synthetic line at 2132.5~nm that is clearly not observed in detector 2 (top right). Even with the majority of the mismatched lines removed the detection of the companion was still unsuccessful.

For detectors 1 and 2 it appears that the synthetic spectra contain many more deeper lines than observed. For detector 3 the red half of the detector was masked out as there appears to be an offset between the observed lines. With 3--4 lines that appear to be consistently offset from the observation it could be a wavelength calibration issue, although the telluric lines appear to be sufficiently corrected in this region, attesting for the quality of the wavelength calibration, and making it incompatible with the offset. For detector 4 the observed lines do not agree at all with the models. With many observed lines not in the model and only one line with some agreement in wavelength, detector 4 is masked out completely and not used in the \(\chi^2\) fit. Individual inspection of the \(\chi^2\) results for each detector also revealed that there was a large discrepancy between the 4th detector and the other three, with a different RV value for the host star and a \(\chi^2\) values an order of magnitude higher. The edge of a deep Hydrogen line (Brackett-\(\gamma\)) off the edge of the detector 4 is also clearly seen in the continuum of the model >2162~nm. 

We applied this same method to the remaining targets, with similar results. In brief, we conclude that the companion spectra cannot be correctly detected in our data using this method.

\begin{figure*}
    \centering
    \includegraphics[width=0.8\hsize]{images/HD211847_result_pcolors.pdf}
    \caption{\(\chi^2\) result grid for observation 2 of {HD 211847}, similar to Figs.~\ref{fig:Mdwarf_contours} and~\ref{fig:HD211847_simulated_contours}. The top right plot shows the application of a single component model (\(C^1\)) while the other three are using a binary model (\(C^2\)). Both left hand panels show the distribution of host temperature and host RV. The top right panel shows the distribution for host and companion temperature, and the bottom right the companion temperature and radial velocity. The red circles indicate the literature values or calculated parameters for the target while the yellow star indicates the minimum \(\chi^2\) solution. The error bar on the \(T_{\textrm{eff}_1}\) is from the literature while the error bars on \({rv}_1\) and \({rv}_2\) are calculated by propagating the orbital parameter uncertainties though the radial velocity equation. The white line shows a 3-\(\sigma\) confidence level about the minimum \(\chi^2\) solution grid point, not always visible here due to the large \(\chi^2\) values.}
    \label{fig:HD211847_result_contours}
\end{figure*}



\begin{figure*}
    \centering
    \includegraphics[width=0.8\hsize]{images/visualize_result_residuals.pdf}
    \caption{Comparison between the observed {HD 211847} spectrum (blue) and the best fit synthetic binary model (orange dashed) for each detector. The bottom section of each panel shows the residuals between the parts of the observation used in the \(\chi^2\) fit and recovered binary model (\(\rm O-C^2\)) in purple. The red dashed line shows the difference between the recovered binary model and the binary model with the exact same parameters except for the estimated companion temperature of 3200~K (\(\rm C^2[3200K]- C^2\)). The grey shading indicated the wavelength regions where masking has been applied. The thinner masked regions that match with cuts in the observed spectra are where the centres of deep (>5\%) telluric lines that have been masked out are.}
    \label{fig:visualinspection-hd2118471}
\end{figure*}


\subsection{Companion injection-recovery}
\label{subsection:injection-recovery}
To determine the detection limits for this method we employ an injection-recovery approach. We take the observed spectra and inject onto them a synthetic companion, at the absolute flux ratio to which it would have been added to a synthetic host with the same parameters. The injected companion RV is set to 100\kmps{} so that the companion lines are well separated from the lines of the host. This separation chosen is slightly larger than what we have with our observations, \(rv_2\) given in Table~\ref{tab:observations}.

We restrict the search space by fixing the host parameters \(T_{\textrm{eff}_1}\) and \(\rm logg_1\) to those recovered fitting the non-injected spectra by a single component model. The wavelength masking is used to reduce the level of mismatch between synthetic and observed spectra. 

We apply the recovery method developed above on the injected spectrum, leaving only the companion \(T_{\textrm{eff}_2}\) and \({rv}_2\) parameters free, to recover the injected companion. We repeated this for injected companions with temperatures below 5000~K. 

We also perform the injection-recovery with synthetic host spectra, representing each target. The wavelength range of the synthetic spectra used for this is three sections interpolated to 1024 values in the wavelength span of detectors 1, 2, and 3. For each section, Gaussian noise is added at the level measured in the corresponding detector for the in the observation of the target being represented.

In Fig.~\ref{fig:injection-recovery} we show the results of the injection-recovery on {HD 30501}. The blue dots represent the recovered companion temperature when injected into real observations, while the orange triangles represent injection into a synthetic host. Error bars of \(\pm100\)~K are included to indicate the grid size, and do not come from the recovery itself. The black dashed diagonal is the temperature 1:1 relation, where a correctly recovered companion should lie.

The grey shaded region indicates the \(\pm1000\)~K temperature range explored for the injection-recovery of the companion. This shows how the bounds of the grid are recovered at low temperatures.

For {HD 30501} the injection onto synthetic and observed spectra produce similar results. At temperatures above 3800~K in both the real and synthetic the injected companion is recovered within 100~K. For injected companion temperatures below 3800~K the temperature recovered is systematically higher than the injected value. This indicates that the companion is not correctly recovered and is affected by the added noise. We determine this temperature to be the upper temperature limit for the recovery. For the other stars we could not conclude on the upper limit due to spectral mismatch issues. In these cases we use the results from the synthetic injection to derive a temperature recovery cut-off for each target, each simulated with the closest host star spectrum.

In Fig.~\ref{fig:injection_shape} we show the minimum \(\chi^2\) for each companion temperature in the recovery grid. We do this for 7 different injected companion temperatures between 2500 and 4500~K. For the higher temperature companions, the \(\chi^2\) is parabolic in shape, recovering the correct temperature, as expected. At lower temperatures there is a strong asymmetry in the \(\chi^2\) with it flattening out on the lower temperature side. 
The 1-, 2-, 3-\(\sigma\) values (with 2 degrees of freedom) of 2, 6 and 11 above the minimum \(\chi^2\) are not shown in the bottom panel of Fig.~\ref{fig:injection_shape} which is a close-up around the minimum \(\chi^2\) as are indistinguishable in the top panel due to the \(\chi^2\) y-scale. The black vertical line indicates the 2300~K temperature limit of the PHOENIX-ACES models.


\begin{figure}
    \centering
    \includegraphics[width=0.95\hsize]{images/inject_recovery_hd30501.pdf}
    \caption{Result of simulated injection-recovery of synthetic companions on {HD 30501}. The blue dots and orange triangles indicate the recovered companion temperature for the observed and synthetic spectra respectively. The \(\pm100\)~K error bars are the grid step of the synthetic models. The black dashed diagonal shows the 1:1 temperature relation. The grey shaded region indicates the \(\pm1000\)~K temperature range explored. Gaussian noise added to the synthetic spectra was derived from the observed spectra.}
    \label{fig:injection-recovery}
\end{figure}


\begin{figure}
    \centering
    \includegraphics[width=0.95\hsize]{images/chi2_shape_investigation_sigmas.pdf}
    \caption{(top) Companion temperature verses \(\chi^2\) for simulations with different injected companion temperatures. Other fixed parameters for these fully synthetic simulations was \(T_{\textrm{eff}_1}=5200\)~K, \(\textrm{logg}_1=4.5\), \(\textrm{logg}_2=5.0\), and both [Fe/H] = 0.0. A fixed Gaussian noise corresponding to a SNR of 300 was used.
        (bottom) A close up view of \(\chi^2\) < 15. The three horizontal grey lines indicate the 1, 2, 3 sigma with 2 degrees of freedom. The vertical dotted lines indicate the location of the minimum \(\chi^2\) recovered for each companion. The black solid vertical in both panels shows the 2300~K cut-off of the PHOENIX-ACES models}
    \label{fig:injection_shape}
\end{figure}

%!TEX root = ../nir_companions.tex
\begin{table}
  %  \begin{threeparttable}
        \centering
        \caption{Upper mass limits of target companions assuming a companion logg = 5.0. Masses are derived from \citet{baraffe_new_2015} evolutionary models using  \(T_{\textrm{eff}}\) and logg. The flux ratio \(\rm F_2/F_1\) is the absolute flux ratio between the cutoff temperature and the target host star.}
        \begin{tabular}{l c c c}
            \toprule
            Target & \(T_{\textrm{eff}}\) cutoff (K) & \(\rm F_2/F_1\) & Mass limit ($\rm M_{jup}$)\\
            \midrule
            {HD 4747}     &  3900 & 0.084 & 598\\ 
            {HD 162020} & 3900 & 0.147 & 598\\
            {HD 167665} & 3800 & 0.054 & 560\\
            {HD 168443} & 4000 & 0.094 & 618\\
            {HD 202206} & 3900 & 0.075 & 598\\
            {HD 211847} & 3900 & 0.079 & 598 \\
            {HD 30501}   & 3800\tnote{a} & 0.106 & 560\\
            \bottomrule
        \end{tabular}
        \begin{tablenotes}
                \item [a] {From observed spectra}
        \end{tablenotes}
  %  \end{threeparttable}
    \label{tab:mass_limits}
\end{table}




Using the temperature cut-off values, we derive an upper mass limit for the companions around our stars using the~\cite{baraffe_new_2015} evolutionary models, finding the closest point matching the spectral temperature cut-off and \(\rm logg=5.0\). These values are given in Table~\ref{tab:mass_limits} and are between 560 and 618 \(\rm M_{Jup}\). The flux ratio between the cut-off and the host star are also provided for, being between 5 and 15\% in this wavelength span.


\section{Discussion}
\label{sec:discussion}
The spectral differential and the synthetic recovery methods attempted here were both unsuccessful in a detection of the BD companion spectra.  The upper mass limits of \(600^{+20}_{-40} \) we set for these companions is very high, roughly six times higher then the BD mass limit \(\sim\)80-90~\(\rm M_{Jup}\).
We discuss potential reasons and solutions for these poor results below, list the lessons learned in this exploratory study {\rd of this dataset}, and provide some guidance for any future attempts with these methods.


\subsection{Synthetic recovery limitations}
\label{subsec:limitations}
In this section we discuss some of the limitations from this synthetic binary method and some options to overcome some of these.

\subsubsection{Mismatch in synthetic models}
\label{subsubsec:mismatch}
We believe the spectral mismatch between the observation and synthetic spectra is the main cause of the unsuccessful companion detection with the $\chi^2$ method with several strong lines in the model not observed in the spectra. This impacts the recovery in two ways; the spectral mismatch causes the \(\chi^2\) values to be large in general, but also causes the companion temperature to be pushed to higher temperatures, up to the constraints allowed by the grid. 

In our examples the logg and metallicity of the synthetic models are held fixed, leaving only temperature to vary. The temperature impacts the synthetic spectral models in two main ways: the flux level of the continuum; and the number and strength of the absorption lines. In the binary model the contributions from the individual components is scaled by the flux ratio. If the temperature of the companion increases then the flux and radius of the companion increases. The contribution of the companion to the binary model increases and the flux ratio \(\rm F_1/F_2\) decreases. This effectively makes the lines of the spectrum of the host component relatively smaller in the normalized binary model spectrum. Due to the large initial mismatch of synthetic spectral lines of the host, a decrease in relative strength of the host lines decreases the \(\chi^2\) value, and is a better ``match'' to the observation. This causes the recovered temperature of the companion to be much higher than expected. If the companion temperature grid is allowed to extend it will recover a companion with a temperature >2000~K above the expected temperature. The \(\chi^2\) approach is dominated by reducing the mismatch in the spectrum of the host rather than actually detecting the spectra of the companion. When preforming the simulations  in Sect.~\ref{subsec:simulated_binaries} there is no spectral mismatch between the simulation and the models, hence they do recover the correct host spectra and get closer values for the companion. 

{\rd Other works have also indicated regions or specific lines in which synthetic models did not reproduce all of the spectral features seen in stellar spectra \citep[e.g.][]{bonnefoy_library_2014, passegger_spectroscopic_2016, rajpurohit_spectral_2016}, although not to the extent observed here.}

\subsubsection{Line contribution of faint companions}
\label{subsubsec:line_contributions}
We calculate the line depths of the synthetic companion spectra to determine the SNR levels required to detect the lines of the binary companions.
One thing easy to overlook when attempting to detect the binary companion at low flux ratios is the actual contribution of the spectral lines of the companion. 
The flux ratio of the continuum for our most promising target is \(\rm F_2/F_1\)\(\sim\)3\% with the other targets having an expected flux ratio around 1\%, and some well below. The spectral lines of the individual components which are the features we are trying to detect with the binary model, have depths on average around 10--20\% of their respective continua,  at-least between 2110--2160~nm. In effect, the companion line features have a depth \(\ll\) 1\% relative to the continuum of the combined spectra. 

In Table~\ref{tab:line_contributions} we calculate some properties of the spectral lines in the PHOENIX-ACES library between 2110--2160~nm. We count the number of spectral lines (\emph{no.\ lines}) deeper than 5\%, and take the average depth (\emph{avg.\ depth}) of these lines. The contribution \emph{cont.\ depth} of the companion lines to a combined spectrum accounts for the flux ratio between the two components. Here we use a Sun-like host with \(T_{\textrm{eff}_1}=5800\)~K. This simplified combination neglects the continuum shapes of both spectral components and uses the average flux ratios for this wavelength range. The PHOENIX-ACES spectra in the temperature range of 2500--5000~K shown in Fig.~\ref{fig:comp_spectra} can be used to get a visual indication of the line density and depth measured here.

There are more lines >5\% deep for the lower temperature spectra, with 360--460 lines in this wavelength range, to be compared with the 31 deep absorption lines found in a Sun-like spectrum. The average line depth of these lines is also larger than the Sun-like spectrum, around twice as deep. However, when combined, the contribution of the companion lines is 1--2 orders of magnitude smaller than the hosts lines due to the low continuum flux ratios. 

For example, with the synthetic model for the companion of {HD 211847}, the average contributions of lines >5\% become only 0.3\% deep in a binary with the Sun-like spectrum. For a companion with a temperature of 2300~K (the lower PHOENIX-ACES temperature limit) the deepest lines contribute lines around 0.1\%. 

%{\bl We use the contributed line depth values to calculate the SNR level required to have Gaussian noise of the same height and the observed SNR required to achieve equivalent contribution from all N lines of the spectrum, \(\rm \textrm{SNR}_N = \textrm{SNR} /\sqrt{N}\). This is for the synthetic spectra which have many more lines than the observed spectra in this wavelength range. }

The SNR of the observed spectra is between 150--350, which is below the SNR of 323 needed for the detection of the low-mass star companion of {HD 211847} with temperature 3200~K and logg 5.0. For our other targets with BD companions at and below the PHOENIX-ACES temperature range, we would need observed SNR >800 to detect the individual spectral lines of the companion. With the SNR increasing with \(\sqrt{N}\) this would require the observational time for each target to be increased by a factor of \(\sim\)10--64.
{\rd This is in line with the recent detection of the spectrum of a non-transiting giant planet by \cite{piskorz_evidence_2016} which utilized nIR spectra with SNR > 2000, from 1-3 hours of observation each.}

Our non-detection of binary companions with low flux ratios is consistent with results from other works. For example~\citet{nemravova_xtauri_2016} performed extensive spectral analysis of a quadruple-star system {$\xi$ Tauri} using 227 spectra in 3 different wavelength bands. Of the four stars in the system they were unable to detect the spectral component of the one which had a luminosity ratio below 1\%. {\rd The secondary detection in optical spectra using spectral matching of KOI was also only able to reach flux ratios of 1\%~\citet{kolbl_detection_2015}.}


%!TEX root = ../nir_companions.tex

\begin{table}
    \small
    \centering
    \begin{threeparttable}[b]

   \caption{Contribution of synthetic lines within 2110--2160~nm of synthetic PHOENIX-ACES spectra to a binary model. \(F_{2}/F_{1}\) is the continuum flux ratio between a spectrum with the given \(T_{\textrm{eff}}\) and logg and a Sun-like spectrum with \(T_{\textrm{eff}}=5800\), logg = 4.5 (right most column). \emph{No.\ lines} is the number of spectral lines deeper than 5\% from the continuum of the individual spectra while \emph{avg.\ depth} is the mean depth of those lines. \emph{Cont.\ depth} is the average contribution, or depth, of these lines in the combined spectrum of a binary with a Sun-like spectrum. The SNR is signal-to-noise level required to have Gaussian noise \(\sigma\) =1/SNR equal to the \emph{cont.\ depth} level in the binary model. All synthetic spectra used here have [Fe/H]=0.0.}
    
    \begin{tabular}{*7c}
        \toprule
        \(T_{\textrm{eff}}\) (K)  & \multicolumn{2}{c}{2300} & \multicolumn{2}{c}{3200} & 5800 (\(\rm F_1\))\\
        logg & 5.0 & 4.5  & 5.0 & 4.5 & 4.5 \\
        \midrule
        \(F_2/F_1\) & 0.006 & 0.019 & 0.029  & 0.091 & 1.000 \\  
        % {>2\%}  & no. lines & 470 &  463 & 414  & 444 & 111 \\
        % & avg.\ depth & 0.20 & 0.23 & 0.10  & 0.12 & 0.04 \\
        % & cont. depth \tablefootmark{a} & 0.0012 &  0.0043 & 0.0028 &  0.0100 &  0.0333\tablefootmark{b} \\ 
        % \midrule
        no.\ lines & 464 & 463 & 365  & 413 & 31 \\
        avg.\ depth & 0.2  & 0.23& 0.11 & 0.12 & 0.10 \\
        cont.\ depth\tnote{a} &  0.0012 & 0.0043 &  0.0031 & 0.0100&  0.0833\tnote{b} \\ 
        SNR  & 833 & 232 & 323  & 100 & 12 \\ 
        %  SNR\(\rm _N\)  & 39 & 11 & 17  & 5 & 2 \\ 
        \bottomrule
    \end{tabular}
 \label{tab:line_contributions}
    \begin{tablenotes}
        \item [a] avg.\ depth \(\times~ \rm F_2 / (F_1 + F_2)\), where \(F_1\) is the component in the far right column.
        \item[b] avg.\ depth \(\times~ \rm F_1 / (F_1 + F_2)\), where \(\rm F_2\) is for the companion with \(T_{\textrm{eff}}\)=3200, logg=4.5.
    \end{tablenotes}
  \end{threeparttable}
   
\end{table}

\subsubsection{\(\chi^2\) asymmetry} 
\label{subsubsec:chi2_assymetry}
In Fig.~\ref{fig:injection_shape} we showed that the shape of the recovered \(\chi^2\) becomes asymmetric when dealing with companion temperatures below around 3800~K. A visual inspection of the spectra reveals the likely cause. In Fig.~\ref{fig:comp_spectra} we show the corresponding spectra between 2111--2165~nm. As the temperature decreases the strongest lines become less prominent, disappearing progressively among the other many small lines that appear at lower temperatures. Hence there are no strong companion lines to easily distinguish one temperature from another. In the flatter part of the \(\chi^2\) curves several low temperature companions are equally well fitted to the simulation/observation.

Figures~\ref{fig:injection-recovery} and ~\ref{fig:injection_shape} show different recovered temperatures but both agree above 3800~K. A higher companion temperature is recovered between 2800 and 3800~K, where as in Fig.~\ref{fig:injection_shape} a lower temperature is recovered. This is probably due to a combination of the noise added, and the asymmetries of the \(\chi^2\) lines. Figure~\ref{fig:injection-recovery} uses the noise level from the observed spectrum while Fig.~\ref{fig:injection_shape} has a SNR of 300.
This large asymmetry can also explain the jump observed in the synthetic recovery temperature around 2700~K in Fig.~\ref{fig:injection-recovery}.

The asymmetry also causes an asymmetry in the \(\chi^2\) error bars which can be seen in the bottom panel of Fig.~\ref{fig:injection_shape}. For instance the recovered value and 1-\(\sigma\) error bars on the 3000~K injected companion is \(2800 ^{+20}_{-100} \), with an asymmetric error bar skewed towards lower temperatures.

The bump observed at 5100~K in the \(\chi^2\) curves is due to a discontinuity in the PHOENIX-ACES modelling. The ``reference wavelength defining the mean optical depth grid'' is changed at 5000~K~\citep[][Sect. 2.3]{husser_new_2013}. Care needs to be taken if trying to detect a companion near this temperature.

\begin{figure}
    \centering
    \includegraphics[width=0.95\hsize]{images/companion_spectra.pdf}
    \caption{PHOENIX-ACES spectra for temperatures between 2500 and 5000 K, corresponding to to the same lines in Fig.~\ref{fig:injection_shape}. The flux units are the native units of the PHOENIX-ACES spectrum, (\(\rm erg\,s^{-1}\,cm^{2}\,cm^{-1}\)), and have not been scaled by the stellar radii. All spectra have a \(\rm logg=5.0\) and \(\rm [Fe/H]=0.0\). The vertical dotted lines indicate the edges of the CRIRES detectors.}
    \label{fig:comp_spectra}
\end{figure}

\subsubsection{Component RV separation}
\label{subsubsec:rv_seperation}
Another factor which could contribute to an unsuccessful detection is the RV separation between the host and companion, \(rv_2\). Estimates for our observations are given in the last column of Table~\ref{tab:observations}. If \({rv}_2\) is small compared to the line width, then all the same lines of both components will be blended. This is indeed the case for {HD 4747}, {HD 211847}, and {HD 202206} with expected \(|{rv}_2| < 2\)\kmps{}, {\rd due to poor observational planning}. This may have contributed to the lack of recovery with both components of the binary model attempting to fit the same features. This may even cause correlation between the parameters of the two components. The RV separation of the two components changes with orbital phase. Having multiple spectra of the same target distributed in phase may allow the RV of the spectral components to be better recovered~\citep[e.g.][]{czekala_disentangling_2017, sablowski_spectral_2016}. 
{\rd Similarly \cite{kolbl_detection_2015} were unable to detect companion stars within 10\kmps{} of the host using optical spectra}.

\subsubsection {Wavelength range}
\label{subsubsec:wavelenght_range_limitation}
{\rd The wavelength range for these observations was chosen specifically due to the location of the \textit{K}-band telluric absorption window. This was to reduce telluric contamination present in the spectra intended for the spectral differential technique. The wavelength range is also very narrow (\(\sim\)50~nm) set by the CRIRES instrument. The small number and inconsistent distribution telluric lines made the wavelength calibration method using the telluric lines difficult in some regions (specifically detectors 2 and 3). For the $\chi^2$ fitting of faint companions this narrow wavelength region likely not the best given the small number of stellar  lines and unique spectral features of the companion. limited stellar  the host and choice, with only a small number of stellar lines to get a good fit of the host, and  This wavelength range is likely not the best choice for the proposed study. For example \citet{passegger_fundamental_2016} use 4 different wavelength regions with lines from different species to fit PHOENIX-ACES models to M-dwarf stars while other studies aiming to detect planetary companions choose wavelength regions which contain strong planetary absorption features such as the absorption of CO and H$_2$O near 2.3 $\mu$m \citep[e.g.[]{dekok_detection_2013, brogi_carbon_2014}.
Applying the binary fitting to a different wavelength region with lines more sensitive to stellar parameters for both stars and BDs, as well as using a larger wavelength range (i.e. provided by the cross-dispersion on CRIRES+), should improve the recovery results of the technique presented here. We note that if the wavelength range is increased by taking separate observations at different wavelengths, not covered by a single exposure, then changes in the RV of both components between the different wavelength observations will need to be accounted for.}


\subsubsection{The {BT-Settl} models}
\label{subsubsec:BT-Settl}
We note that the PHOENIX-ACES models are not the only spectral libraries available with the other notable library considered for this work is the {BT-Settl} models, ~\citep{allard_model_2010,allard_btsettl_2013,baraffe_new_2015}. The included modelling of dust and cloud formation, as well as hydro-dynamical modelling atmospheric mixing/settling for atmospheres with \(T_{\textrm{eff}}\) below \(\sim\)2600~K, make the {BT-Settl} models valid across the regime from stars to BDs as cool as 400~K. As the {BT-Settl} models are suitable to model the atmospheres of the brown dwarfs they would be useful for the companion recovery technique developed here. However, as shown in Sect.~\ref{subsection:results-hd211847} and~\ref{subsection:injection-recovery}, we were unable to successfully recover the 155 \(\rm M_J\) (\(T_{\textrm{eff}}\sim\)3200~K) low mass star companion of {HD 211847} and derived a temperature upper limit for our methodology of around 3800~K. These are both well above the 2300~K cut-off of the PHOENIX-ACES models and for the onset of dust- and cloud-formation phenomena, at 2600 K..

Fig.~\ref{fig:hd211847-models} shows again the minimum \(\chi^2\) solution for detector 1 of the second {HD 211847} observation, this time including the {BT-Settl} solution with the same parameters. Although the PHOENIX-ACES and {BT-Settl} models differ slightly they both have a large spectral mismatch to the observations. As such, we did not use the {BT-Settl} models for the simulation and results above as we did not see any special advantage in using them. 

The ease of access to find, download, and use PHOENIX-ACES spectral library, available in the fits file format, compared to older {BT-Settl} libraries is another reason for the current favoured use of the PHOENIX-ACES library.

Although the newer generations synthetic spectral models are improving and match the overall spectral energy distribution reasonably well there are still regions in the H and K band where there is room for improvement~\citet{rajpurohit_spectral_2016}. The spectral mismatch in the region studied here is still too large for spectral recovery of companion brown dwarfs. In the nIR we have compounding problem: the model input physics of sub-stellar temperatures and chemistry combined with the general difficulty of the nIR.

\begin{figure}
    \centering
    \includegraphics[width=0.95\hsize]{images/HD211847_ACES_BTSettl.pdf}
    \caption{Detector 1 spectrum for {HD 211847} (blue) alongside the PHOENIX-ACES (orange dash-dot) and {BT-Settl} (green dashed) synthetic spectra for the host star only, with parameters \(T_{\textrm{eff}}=5700\)~K, \(\rm logg=4.5\) and \(\rm [Fe/H]=0.0\). Both synthetic models have been normalized and convolved to \(\rm R=50\,000\). There is a 0.05 off-set between each spectrum}
    \label{fig:hd211847-models}
\end{figure}


\subsubsection{Impact of logg}
\label{subsubsec:logg}
Logg, a measure of surface gravity, is related to evolutionary state and the size of the star with smaller logg values usually indicating larger radii stars. This parameter has a large impact on the radius and flux ratio of the binary models. In the PHOENIX-ACES models a decrease in logg from 5.0 to 4.5 increases the models effective radius by \(\sim\)1.75 in the temperature range investigated here. This change in radius alone roughly triples (\(1.75^2\)) the absolute flux of the synthetic spectrum, neglecting any changes to the shape of the actual spectrum. Therefore, there are large jumps in the model flux ratios if the logg is allowed to vary, with lower logg values for the companion being favoured as the increased flux ratio decreases the mismatch of the host component to the observations. This large impact of logg on the spectral library absolute flux is one reason for keeping the logg of each component fixed in the \(\chi^2\) results presented in Sect.~\ref{sec:results}.

\subsubsection{Interpolation}
\label{subsubsec:interpolation}
It is common to interpolate between the synthetic spectral grids to fit and derive parameters in between the grid points~\citep[e.g.][]{nemravova_xtauri_2016, passegger_fundamental_2016}. Instead of interpolation~\cite{czekala_constructing_2015} use a spectral emulator to use Principal Component Analysis to create eigenspectra for the synthetic library and Gaussian processes to derive a probability distribution function of possible interpolation spectra to account for uncertainties in the interpolation required for high signal-to-noise spectra.

However, we did not incorporate any interpolation into the companion recovery at this stage. This could be something to be added in the future to refine the recovered parameters, and to help the transition between the grid logg values. Codes are readily available to perform spectral interpolation which could be utilized for this, two of them are \emph{pyterpol}\footnote{https://github.com/chrysante87/pyterpol}~\citet{nemravova_xtauri_2016} and \emph{Starfish}\footnote{https://github.com/iancze/Starfish}~\cite{czekala_constructing_2015}. 


\subsection{Future implementation}
\label{subsec:future}
\subsubsection{High resolution instrumentation}
\label{subsubsec:highres}
The future of high resolution near- and mid- IR spectrographs is looking bright, with many new ground- and space-based instruments currently being developed. Notable examples include CARMENES (550--1710~nm, R=82\,000) which is now operational~\citep{quirrenbach_carmenes_2014}, while SPIRou (980--2350~nm, R=73\,500)~\cite{artigau_spirou_2014} and NIRPS (970--1810~nm, R=100\,000)~\cite{bouchy_nearinfrared_2017} are still being assembled and installed. The eagerly awaited JWST \textbf{cite} will also be launched soon\footnote{Recently pushed to around May 2020} providing observations in both the near-IR (600--5300~nm, R=2700) and mid-IR (4900--28\,800~nm, R$\sim$1550--3250) regions without contamination from our atmosphere.

The upgrade of CRIRES to CRIRES+~\citep{dorn_crires_2016} will increase the wavelength coverage of a single shot capture by at least a factor of 3-5. This larger wavelength span would be extremely beneficial for the \(\chi^2\) performance of the spectral recovery method, increasing the number of useful lines and spectral features to be fitted with the models. 

On the modelling side, there are continual improvements in atmospheric modelling and their associated synthetic spectral libraries: as seen with the evolution of the {BT-Settl} models~\cite{allard_btsettl_2013}. With additional physics and improved line lists and solar abundances~\citep[e.g.][]{asplund_chemical_2009,caffau_solar_2011}, the synthetic libraries are reaching a better agreement with nIR observations. An improved agreement between the nIR observations and synthetic spectra will be crucial to improve the performance of the spectral recovery technique presented here.

Although not successful with the CRIRES data used here, the instrumental stage is set to attempt these techniques presented here using the next-generation of high resolution spectrographs. The lessons learned in this analysis need to be taken into account in order to achieve the best chance of a successful detection. 

\subsubsection{Differential scheduling challenges}
\label{subsubsec:differential-schedualing}
This work has revealed that more care needs to be taken in planning the observations for the spectral differential analysis of faint companions in the future. Paying attention in particular to the FWHM of the lines in the region (governed by resolution and wavelength); the estimated companion \(\Delta RV \); the previous observations from different observing periods; and keeping consistent detector settings.

The original goal for the observations was to obtain two different and ``clearly separated radial-velocities'' for the secondary companion. However, the program was assigned a low-priority (C, in ESO grading) and, possibly due to operational reasons, the original time requirements necessary to secure well separated RVs for the companion spectra could not be met. This meant that all observations were insufficiently separated to extract a differential spectra for the companion.

The long orbital periods of these targets is also a contributing factor to the insufficient separations. Most of the targets observed here have orbital periods much longer than an observing semester (183 days). An optimal pair of observations (achieved at the extrema) would need to have been obtained from separate observing periods (between 2 months and 19 years apart). In some cases, even observations taken at the beginning and end of the semester would not be sufficient to achieve companion separation (depending on the phase). Requiring separate observing periods to even achieve the minimum \(\Delta \rm RV\) larger than the line FHWM.\@At the time it was impossible to ask for time over several semesters in a regular proposal.

Our study demonstrates the importance of proposals for projects that need to be extended over several semesters or years. In the ESO context, this corresponds to ``Monitoring proposals''~\citep[e.g.][pg. 18]{eso_eso_2017}. Observations of the targets explored here, with long orbital periods in particular, would benefit from the ability for multi-period proposals and newer scheduling systems which allow for tighter scheduling constraints, such as a companion RV separation. 

For future observations we suggest that the known orbital solution of the companion be used to estimate the companions' RV curve during the observing period, with the companion \(M_2\sin{i}\) providing an RV upper-limit. Knowing the instrumental wavelength and resolution, a constraint can then be set to avoid taking observations when the companion spectra are insufficiently separated, or \(\Delta RV\) < FWHM.\@This constraint can be set using the absolute and relative \emph{time-link} constraints available in ESO's Phase 2 Proposal Preparation (P2PP) tool.
Additionally, analysing the known orbital solution before-hand, to determine RV constraints will also help identify the best time to observe, if observations from separate periods will be required or, if an optimally separated companion differential is even feasible.

\subsection{Other techniques}
We note that there are many other disentangling techniques to separate mixed spectra of binary systems,~\citep[e.g.][]{hadrava_disentangling_2009}. These require more than two observations, with  \(n+1 \) observations used to set up a system of linear equations to solve for \(n \) spectral components~\citep[e.g.][]{simon_disentangling_1994,czekala_disentangling_2017, sablowski_spectral_2016}. These methods are ideal for many well spaced observations. For example the ideal situation for the SVD method of~\citet{sablowski_spectral_2016} is homogeneous samples of at least half the period, to identify the moving spectral components. 
{\rd Recently the cross-correlation and maximum-likelihood techniques \citep[e.g.][]{lockwood_nearir_2014, piskorz_evidence_2016} have been successful in detecting the faint companion spectra of giant planets using several spectra with very-high SNR (>2000) obtained with longer observational time (1 -- 3 hours) each taken across the full orbital range.} The few, short and insufficiently separated observations we analyse here are not suitable to apply these advanced techniques and are beyond what we have attempted here.


\section{Conclusions}
\label{sec:conclusions}
This work explored two methods to try and detect the faint spectra of stellar companions, using high resolution near-infrared spectra. Two different methods were explored with many limitations uncovered. 

The objective of the observations acquired in this program was to a apply the differential technique. For the differential technique the observations need to be sufficiently separated, such that the RV of the companion is greater than the FWHM to avoid spectral cancellation of the companion. Unfortunately, due to operational reasons, this condition was not met. As such we employed an alternative method, that we termed the spectral recovery, which ended up revealing different difficulties.

For the spectral recovery the host-companion RV separation should also ideally be greater then the FWHM to avoid blended lines. The spectral mismatch between models and reality in the nIR negatively affects the performance of the synthetic recovery technique on the observed spectra. With all these effects we are unsuccessful in the detection of the nIR spectra of BD companions,  with the mass upper-limits set at \(\rm 600~M_{Jup}\) from the synthetic recovery technique.

This work highlights many of the difficulties when dealing with the spectral recovery of nIR spectra. The obstacles to overcome are the data reduction of nIR CMOS detectors, that are not yet at the level of visible CCDs, along with a precise telluric correction and wavelength calibration (two interrelated aspects, as thoroughly discussed). Another important aspect is the mismatch between nIR high-resolution spectra and the observed spectra. In spite of the continuous effort of the modelling community, our work, along with several cited contemporary ones, shows that this mismatch is still one of the main factors preventing us from perform spectral recovery in the nIR. This work highlights that this is a compound problem for Brown Dwarfs, for which the spectral models are worse informed due to lack of observations at high-resolution.

Other than the improvement of the spectral models, the observing community can increase their odds of success by paying attention to the scheduling of observations and the wavelength domains to explore. Our work shows that observing in the areas of lower telluric absorption, as is frequently done, is not a guarantee of success due to the scarcity of deep lines in cold objects. Moreover, due to the mismatch between models and observations, the ability to obtain a first spectra before settling on a wavelength range, or changing settings on the fly, is extremely useful for the success of these campaigns.

We hope that this work can act as a guide for the planning of future observations of targets with faint BD and planetary companions with the upcoming generation of high resolution spectrographs in the near- and mid- infrared such as CRIRES+ and JWST observations.


\section*{Acknowledgements}

This work was supported by Funda\c{c}\~ao para a Ci\^encia e a Tecnologia (FCT) (Portugal) research grants through national funds and from FEDER through COMPETE2020 by the following grants: UID/FIS/04434/2013 \& POCI--01--0145-FEDER--007672, PTDC/FIS-AST/1526/2014 \& POCI--01--0145-FEDER--016886, and PTDC/FIS-AST/7073/2014 \& POCI-01-0145-FEDER-016880.
J.J.N. acknowledges support from FCT though the PhD::Space fellowship PD/BD/52700/2014.
P.F., N.C.S. acknowledge support from FCT through Investigador FCT contracts IF/01037/2013CP1191/CT0001, IF/01312/2014/CP1215/CT0004, IF/00169/2012/CP0150/CT0002, and IF/00028/2014/CP1215/CT0002. P.F. further acknowledges support from FCT in the form of an exploratory project of reference IF/01037/2013CP1191/CT0001.
This work made use of PyAstronomy\footnote{PyAstronomy can be found at \url{https://pyastronomy.readthedocs.io}} and many other open source and under-credited software packages numpy, scipy, astropy, Starfish to name a few.
This research has made use of the SIMBAD database, operated at CDS, Strasbourg, France.


%%%%%%%%%%%%%%%%%%%%%%%%%%%%%%%%%%%%%%%%%%%%%%%%%%

%%%%%%%%%%%%%%%%%%%% REFERENCES %%%%%%%%%%%%%%%%%%

% The best way to enter references is to use BibTeX:
\bibliographystyle{mnras}
\bibliography{nir_companions}

%%%%%%%%%%%%%%%%%%%%%%%%%%%%%%%%%%%%%%%%%%%%%%%%%%

%%%%%%%%%%%%%%%%% APPENDICES %%%%%%%%%%%%%%%%%%%%%

\appendix

    
    \section{Direct Subtraction Method}
    \label{appendix:A1}
    Here we outline the direct subtraction method used, which is similar to previous works~\citep{ferluga_separating_1997,kostogryz_spectral_2013}. Assuming that the instrumental profile and atmospheric absorption are dealt with appropriately, the spectrum received from the host-companion pair is given by the superposition of two spectral components (\(J_{1} \), \(J_{2} \)), where \(\lambda-v\) represents the Doppler shift \(\lambda(1-v/c)\) by velocity \(v\).
    \begin{equation}
    \textrm{I}(\lambda) = \textrm{J}_{1}(\lambda - v_{1}) + \textrm{J}_{2}(\lambda - v_{2})
    \end{equation}
    or shifted to the host stars rest frame,
    \begin{equation}
    \textrm{I}(\lambda + v_{1}) = \textrm{J}_{1}(\lambda) + \textrm{J}_{2}(\lambda - v_{2} + v_{1})
    \end{equation}
    Here, the subscripts 1 and 2 indicate the spectrum of the host and companion respectively, and \(\lambda \) represents the wavelength of the spectra.
    The spectra of the host star (\(\textrm{J}_{1} \)) is removed though the mutual subtraction of the host spectrum from two separate observations, denoted with subscripts \(a\) and \(b\), correcting for RV motion of the host star.
    \begin{align}
    S(\lambda) &= \textrm{I}_{a}(\lambda + v_{1a}) - \textrm{I}_{b}(\lambda + v_{1b}) \nonumber \\
    &= \textrm{J}_{2}(\lambda - v_{2a} + v_{1a}) - \textrm{J}_{2}(\lambda - v_{2b} + v_{1b}) \nonumber \\
    %  &= \textrm{J}_{2}(\lambda - v_{2a}) - \textrm{J}_{2}(\lambda - v_{2b} - v_{1a} + v_{1b}) \nonumber \\
    S(\lambda + v_{2a}) &= \textrm{J}_{2}(\lambda) - \textrm{J}_{2}(\lambda + \Delta RV ) \label{eqn:sprofile}
    \end{align}
    where,
    \begin{equation}
    \Delta RV = v_{2a} - v_{2b} - v_{1a} + v_{1b} \label{eqn:k}
    \end{equation}
    
    is the actual RV difference (\(\Delta RV \)) between the two companion spectra.
    
    The resulting differential spectra, dubbed \emph{s-profile} by~\citet{ferluga_separating_1997}, is composed of just the companion spectra, shifted and subtracted from itself.
    
    From binary dynamics~\citep[e.g.][]{murray_keplerian_2010} the RV amplitudes of the host and companion are related through the mass ratio, \(q \), while having an opposite sign.
    \begin{align}
    v_{2}  &= -\textrm{q} * v_{1} \label{eqn:q_relation}
    \end{align}
    This equation was used to calculate the expected companion RV for each observation in Table~\ref{tab:observations}. 
    
    We can simplify Eq.~\ref{eqn:k} by expressing it in terms of the host RV and mass ratio,
    \begin{align}
    k &= -q v_{1a} + q v_{1b} + v_{1a} - v_{1b} \nonumber \\
    &= (1 - \textrm{q})(v_{1a} - v_{1b}). \label{eqn:k_simplified}
    \end{align}
    If we are able to determine the \(\Delta RV \) between the the companion spectra, k, from the s-profile~\citep[see ][]{ferluga_separating_1997} then we can determine the mass ratio of the system, q, thereby constraining the mass of the companion. The values \(v_{1a} \) and \(v_{1b} \) are calculated from the orbital parameters of the system and are the same values already used to shift and mutually cancel the host spectrum.
    
    This method is very similar to~\citet{kostogryz_spectral_2013} except that they focus on M-dwarfs host stars with the observations taken at the extrema, in which the companion lines are well separated.
    
    \subsection{Estimating parameters of observations}
    To estimate the  differential amplitude signal we estimated the expected parameters of the impact of the 
    
    \subsection{Results of spectral differential analysis}
    
    \label{appendix:A2}
    We applied the spectral differential procedure outlined above on the wavelength-calibrated and telluric-corrected CRIRES observations. The spectra were corrected for Earth's barycentric RV using the \emph{helcor} PyAstronomy\footnote{https://pyastronomy.readthedocs.io} function ported from the REDUCE IDL package (See~\citet[][]{piskunov_new_2002}) and Doppler shifted to the rest frame velocity of the system. The spectra were then subtracted from each other and analysed as described above.
    
    It is necessary to have a consistent instrumental setup~\citet{ferluga_separating_1997}, to avoid introducing extra instrumental effects (e.g.\ slit-width and/or filters) into the spectral differentials and to always observe the same wavelength range and maximize the information to be extracted. In our case, the second observation of {HD 202206} and fourth of {HD 30501} were taken with different filters compared to the other observations. Therefore, these two observations could not be used for this differential analysis. As noted in~\citep{hadrava_disentangling_2009}, any spectral differences in the filters would add extra unknown signal/noise making it harder to disentangle the faint spectral differences.
    
    
    \begin{figure}
        \includegraphics[width=0.95\hsize]{images/differential.pdf}\\
        \caption{ (Top) A reduced CRIRES observation of {HD 30501} (blue) for detector 1 between 2112--2124~nm along with the tapas telluric absorption model ({orange} dashed) used for the wavelength calibration and telluric correction. (Middle) The telluric corrected spectra. (Bottom) ({blue}) Differential spectra for {HD 30501} between observations 1 and 3. ({orange} dashed) Simulated ``perfect'' differential using PHOENIX-ACES spectra with parameters \(T_{\textrm{eff}} = 2500~K \), \(\rm logg=5.0\), and \(\rm [Fe/H]=0.0 \), with the same \(\Delta RV \) as the observations. The shaded regions indicate where the telluric {green} and host star {blue} spectra are > 4\% deep.}
        \label{fig:spectral_example}
    \end{figure}
    
    We performed the differential analysis for all targets but only show our most favourable case here, {HD 30501}, because it is the second largest companion in our sample at \(\rm 90~M_{Jup}\) and also has the second largest RV separation between observations. The differential spectra recovered for {HD 30501} is shown at the bottom panel of Fig.~\ref{fig:spectral_example}. The presence of deep (\(>4\% \)) stellar and telluric lines in the original spectrum is shaded by the blue and green regions respectively. This indicates that the features of the differential spectrum near these shaded regions are likely due to imperfect telluric correction and host mutual cancellation.
    The mutual cancellation of the stellar host works well for the \(\sim\)40\%  deep line near 2117~nm, being completely removed, but it does not do so well for the smaller \(\sim\)10\% deep line around 2121.5~nm. The residual for the large \(\sim\)40\% deep telluric line near 2118.5~nm is quite prominent. There is also a wider residual due to three neighbouring lines \(\sim\)10\% deep around 2120~nm which cause features in the differential spectrum. One possible explanation is that the continuum normalization near 2120~nm was influenced by this grouping of lines.
    
    To understand the observed differential signal we simulated a differential spectrum of {HD 30501} using a synthetic PHOENIX-ACES spectra with parameters \(T_{\textrm{eff}} = 2500~K \), \(\rm logg=5.0 \), and \(\rm [Fe/H]=0.0 \), with a RV offset estimated from the observation times. These parameters represent an estimated companion \(T_{\textrm{eff}} \) with the metallicity and logg similar to the host (closest grid model). The model spectra were convolved to \(\rm R=50\,000 \), continuum normalized and scaled by the estimated flux ratio of the companion. We do not include any synthetic host or telluric spectra and as such simulate the differential result of a ``perfect'' host cancellation with no telluric contamination present. This is the ideal-case scenario, and we stress that it is impossible to simulate the effect of improper telluric correction in a meaningful way. When comparing the simulated and observed differential in the bottom panel of Fig.~\ref{fig:spectral_example}, there is a striking amplitude difference. The orange-dashed line of the simulated differential spectrum amplitude is of a much smaller scale than the observed differential. This demonstrates that the amplitude of the differential signal we are trying to detect is much smaller than the residuals created by this differential technique. 
    
    The amplitude of the differential signal is lower than we expected due to the very low \(\Delta RV\) between the observation pairs. The maximum \(\Delta RV\) between observation pairs, for the observations investigated in this work, are provided in Table~\ref{tab:estimatedparameters}. {\rd There is no \(\Delta RV\) for {HD 4747} as there was only a single observation. We also provide the phase coverage for our targets. We calculate this as the ratio of time between the observed pairs and the orbital period, and show that the fraction of the orbit covered is very small, all except one covering less than 1 percent of the orbit.}
    
    In our best case, {HD 30501}, the \(\Delta RV\) of the companion between observations is 1.346\kmps{}. For comparison, a single Gaussian absorption line, to be shifted by \(\Delta\lambda = \rm FWHM\) would need a \(\Delta RV\) of \(v_{\textrm{FWHM}} = c/R =~\sim\)6\kmps{}. Since the \(\Delta RV\) are shifted by a smaller value than the FWHM, the spectral lines of the reconstruction mutually cancel themselves, diminishing the amplitude of the differential signal significantly. As the companion spectra are already faint (with a flux ratio at the percent level) the differential signal is not detectable within these observations and noise level.
    When the \(\Delta RV \) of the companion is smaller than the FWHM of a line there is a mutual subtraction of the companion spectra, diminishing the detected amplitude of the differential signal, and removing the ability to detect the companions using this method. Observations need to be spaced further apart in time/phase to achieve a larger \(\Delta RV\) separation and increase the amplitude of the differential. Of course once there is a separation there will be complex interactions between neighbouring lines that need to be accounted for.
    
    \subsection{Relative differential amplitude}
    To probe this issue further we investigated how the amplitude of the differential signal changed with \(\Delta RV \). Taking the same PHOENIX-ACES spectra (\(T_{\textrm{eff}} = 2500~K \), \(\rm logg=5.0 \), \(\rm [Fe/H]=0.0 \)), we computed differential spectra for a range of \(\Delta RV\)s between \(\pm10\)\kmps{}. Figure~\ref{fig:diff_amp} shows how the relative amplitude of the differential spectrum in the wavelength range 2110--2123~nm (detector 1 of our CRIRES observations) changed as the spectra are offset. At each RV step we take the maximum absolute differential value found. These are then normalized by the median value outside of the line FWHM (dashed vertical lines), between \(\pm(7-10)\)\kmps{}, to give a relative differential amplitude, independent from the depth of a specific line. For comparison we also show the relative amplitude of the differential spectrum for a single Gaussian and Lorentzian line when Doppler shifted by the same \(\Delta RV \)s. The shape of these results is also consistent with the analytical form of the differential spectra~\citet[][eqn.~A.1]{ferluga_separating_1997}.
    
    At a \(\Delta RV\) difference of zero, spectral lines of the companion completely cancel each other out, resulting in zero amplitude. As the RV separation increases, the individual lines stop cancelling themselves until a maximum differential amplitude is achieved when the lines are fully separated from themselves (equal to the line depth). 
    
    The two solid vertical lines in Fig.~\ref{fig:diff_amp} indicate the estimated \(\Delta \textrm{RV}\)=1.346\kmps{} separation for our best target, {HD 30501} from Table~\ref{tab:estimatedparameters}, given known orbital parameters and the observation times. This shows that our differentials have severely reduced amplitude, \(<20\%\) relative to well separated individual lines. As the companion spectra are faint and in combination with a host star at 1\% flux ratio the >80\% extra reduction in signal amplitude makes this detection impossible with these observations.
    
    In the synthetic spectrum (and of course real spectra) neighbouring spectral lines begin to interfere, leading to an impact on the measured relative amplitude. We suspect that the interaction of neighbouring lines is one possible cause for the difference between the theoretical and simulated shape between 2 and 6\kmps{}. Beyond the RV range present the amplitude becomes complicated due to neighbouring line interaction, but as the \(\Delta RV\) for all our spectra fall well short of this region we did not investigate this further.
    
    \begin{figure}
        \includegraphics[width=0.95\hsize]{images/relative_amplitude.pdf}
        \caption{Simulated relative amplitude of differential spectra at different RV separations revealing the diminished amplitude at small orbital separations. The solid blue line shows the maximum relative amplitude of the differential signal (from a shifted copy of itself) of a PHOENIX-ACES spectrum with \(T_{\textrm{eff}}=2500~K, \textrm{logg}=5.0, \textrm{[Fe/H]}=0.0\), in the wavelength region 2110--2123~nm. The maximum difference is normalized by the average amplitude between 7--10\kmps{}, representing a complete line separation. The orange (dashed) and green (dot-dashed) lines represent the relative amplitude of a differential spectrum for a single Gaussian and single Lorentzian absorption line respectively, each with a unitary amplitude and a \(\rm FWHM = \lambda / R \). Beyond the FWHM lines the difference of the synthetic spectrum becomes more complicated due to the interaction of neighbouring lines. The solid vertical lines indicate the estimated companion \(\Delta RV\) in these observations while the dashed vertical lines indicate the RV corresponding to the FWHM at this wavelength and resolution. Beyond the FWHM RV the synthetic spectrum becomes complicated due to the interaction of neighbouring lines.}
        
        \label{fig:diff_amp}
    \end{figure}
    
    % Table of estimates of parameters
    %!TEX root = ../nir_companions.tex
\begin{table*}
  %   \begin{threeparttable}[b]
        \small
        \centering
        \caption{Estimated flux ratios, orbital semi-amplitude and RV separation of the companion, given the companion mass (\(\textrm{M}_{2}\) or \(\textrm{M}_{2} \sin{i}\)) from Table~\ref{tab:orbitparams} and observation times from Table~\ref{tab:observations}.} 
        \begin{tabular}{l | c c c c c c c | c c c}[hb]
            \toprule
            & Host& Companion &  Estimated  & Estimated &  Estimated & Estimated &  &    \\  % 2017
            Companion     & M$_{K}$& M$_{K}$& \(\rm F_{2}/F_{1} \)   & \(\rm N_{2}/N_{1} \) (noise ratio) & \(\rm K_2\) &   \(\Delta RV\) & Phase coverage \\
            & & & \textit{K}-band     & & (\kmps{}) & (\,ms\(^{-1}\)) & (\%) \\
            \midrule
            {HD 4747}        & & & \(3\times10^{-4} \)   & 76 &  -10.65 & -  &  -  \\  % 2017
            {HD 162020}   & & & \(7\times10^{-7} \)   & 1615  &  -98.92\tnote{a} &  2344.24     & 0.28~~  \\  %
            {HD 167665}    & & & \(2\times10^{-4} \)   &  105    &  -14.47\tnote{a}  &   138.45     & 0.18~~  \\  %  -- \(2\times10^{-5} \)  best case based on age rage.
            {HD 168443b} & & & \(1\times10^{-16} \)  &    \(1\times10^{8} \)   &  -64.65\tnote{a}&   257.16   & 0.035 \\ 
            {HD 168443c} & & &  \(1\times10^{-11} \)  &   \(4\times10^{5} \)     &  -18.05\tnote{a}  &   0.95   &  0.001 \\  %(c)
            {HD 202206}B  & & & \(8\times10^{-7} \)  &   1586 &  -6.79 & 145.17   & 0.74~  \\  %(B)   % May2017
            {HD 202206}c  & & &  \(5\times10^{-15}\)   &     \(2\times10^{7} \) &   -2.50     &   0.67     &  0.15~  \\  %(B)   % May2017
            {HD 211847}B  & & &  0.01 &  14   & $-$1.85 & 3.88   & 0.09~  \\  %B % 2017
            {HD 30501}      & & &  0.002  &  27  &  -16.12    &  1346.46      & 5.8~~  \\
            \bottomrule& & 
            \end{tabular}
    \begin{tablenotes}
        \item[a] {Maximum \(K_2\) only given \(M_2 \sin{i}\)}
      \end{tablenotes}
%    \end{threeparttable}
    \label{tab:estimatedparameters}
\end{table*}
    
    
    \section{Estimating companion parameters}
    In this appendix we detail how we calculate RV of the components and determine the flux ratio of our targets given their literature masses  \(M_2\) or \(M_{2}\sin{i}\). We refer to these calculations as estimates as in some cases we only have the companions minimum mass.
    
    \subsection{RV equation}
    We use the Keplerian orbit RV equation to estimate the RV of the host and companions at the time of each observation, \(t\): 
    \begin{equation}
    \label{eq:rv_equation}
    RV = K [\cos{(\nu(t) + \omega)} + e\cos{(\omega)}]
    \end{equation}
    Here, \emph{K} is the \emph{semi-major amplitude}, \(\nu\) is the \emph{true anomaly}, \(e\) is the orbital \emph{eccentricity}, and \(omega\) is the \emph{argument of periastron}. The true anomaly is not only a function time, \(t\), but also the orbital period \(P\),  the \emph{time of periastron passage}, \(T_0\), and eccentricity. The literature parameters for each target are provided in Table~\ref{tab:orbitparams}.
    To determine the RV of the companion we transformed the RV semi-amplitude of the host \(K_{1} \) into the semi-amplitude for the companion \(K_{2} \) using the mass ratio,
    \begin{equation}
    \label{eqn:mass_ratio}
    q = \textrm{M}_{2} / \textrm{M}_{1} = \textrm{K}_{1} / \textrm{K}_{2}
    \end{equation}
    
    We note that for the targets in which only the minimum mass (\(\textrm{M}\sin{i} \)) is known, this equation will indicate the maximum \(K_2\) value for the companion. The estimated \(K_2\) for each companion is provided in Table~\ref{tab:estimatedparameters} while the RV for both components at the time of each observation is provided in Table~\ref{tab:observations}.
    
    The error on estimated RV values, shown in Fig.~\ref{fig:HD211847_result_contours} is calculated by applying the general error propagation formula~\citep{ku_notes_1966} and using the  errors on the orbital parameters. For a function, \(f\), with errors on the inputs \(\delta x\), \(\delta y\) etc., it follows: 
    \begin{align}
    f &= f(x, y, z, \ldots)\\
    \delta f &= \sqrt{{\left( \frac{\partial f}{\partial x} \delta x\right)}^2 + {\left(\frac{\partial f}{\partial y} \delta y\right)}^2 + {\left(\frac{\partial f}{\partial z} \delta z\right)}^2 + \ldots}
    \end{align}  
    
    \subsection{Estimating Companion-host Flux ratio}
    \label{compaion flux ration}
    The companion-host flux or contrast ratio of the systems are calculated using \( \frac{F_{2}}{F_{1}} \approx 2.512^{m_{1}-m_{2}} \), where \(m_{1} \) and \(m_{2} \) are the magnitude of the host and companion respectively. For this work we specifically use the magnitudes in \textit{K}-band. The magnitudes of the hosts, \(m_{1} \), are obtained from SIMBAD~\citep{wenger_simbad_2000} while the magnitudes for the companions, \(m_{2} \), are estimated from stellar evolution models~\citet{baraffe_evolutionary_2003, baraffe_new_2015}. The stellar evolution models are interpolated to the companion mass  (or \(M_{2}\sin{i}\)) and system age, then the magnitude in the \textit{K}-band magnitudes extracted. The \textit{K}-band magnitudes, used for the host and companions along with the calculated flux ratio for each target is given in Table~\ref{tab:estimatedparameters}. For the companions in which only the minimum mass is known then the flux-ratio will be the lower limit, or worst case scenario.
    
    These evolution models also provide a first estimate of the companions other properties such as \(T_{\textrm{eff}}\), logg, \(R/R_{\sun}\) and the magnitude in many other wavelength bands.
    
    The companion \(T_{\textrm{eff}}\) and logg values specifically were utilized to influence the selection of synthetic model grids to perform the \(\chi^2\) analysis over in Section~\ref{sec:results}.
    
    A simple tool\footnote{Available at \url{https://github.com/jason-neal/baraffe_tables}} was created to calculate the flux ratio using the~\citep{baraffe_evolutionary_2003,baraffe_new_2015} evolution tables. Given the host star name, the companion mass and a stellar age it interpolates the available Baraffe tables to the companion mass and age specified. The host's name is used to query the {SIMBAD} database to obtain stellar properties, specifically magnitudes, to calculate the flux ratios. It can also work in reverse to estimate a companion mass when provided with a flux ratio.

   A noise ratio between the host and companion is also calculated via \(N_{2}/N_{1} = \sqrt{2} \times\sqrt{F_{1}/F_{2}}\) and given in Table~\ref{tab:estimatedparameters}. 
   
%%%%%%%%%%%%%%%%%%%%%%%%%%%%%%%%%%%%%%%%%%%%%%%%%%


% Don't change these lines
\bsp{}	% typesetting comment
\label{lastpage}
\end{document}

% End of mnras_template.tex